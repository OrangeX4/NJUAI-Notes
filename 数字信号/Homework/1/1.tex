\documentclass[a4paper,UTF8]{article}
\usepackage{ctex}
\usepackage[margin=1.25in]{geometry}
\usepackage{color}
\usepackage{graphicx}
\usepackage{amssymb}
\usepackage{amsmath}
\usepackage{amsthm}
\usepackage{enumerate}
\usepackage{bm}
\usepackage{hyperref}
\usepackage{pgfplots}
\usepackage{epsfig}
\usepackage{color}
\usepackage{tcolorbox}
\usepackage{mdframed}
\usepackage{lipsum}
\usepackage{framed}
\usepackage{setspace}

\newmdtheoremenv{thm-box}{myThm}
\newmdtheoremenv{prop-box}{Proposition}
\newmdtheoremenv{def-box}{定义}

\setlength{\evensidemargin}{.25in}
\setlength{\textwidth}{6in}
\setlength{\topmargin}{-0.5in}
\setlength{\topmargin}{-0.5in}
% \setlength{\textheight}{9.5in}
%%%%%%%%%%%%%%%%%%此处用于设置页眉页脚%%%%%%%%%%%%%%%%%%
\usepackage{fancyhdr}                                
\usepackage{lastpage}                                           
\usepackage{layout}                                             
\footskip = 10pt 
\pagestyle{fancy}                    % 设置页眉                 
\lhead{2022年秋季}                    
\chead{数字信号处理}                                                
% \rhead{第\thepage/\pageref{LastPage}页} 
\rhead{作业一}                                                                                               
\cfoot{\thepage}                                                
\renewcommand{\headrulewidth}{1pt}  			%页眉线宽,设为0可以去页眉线
\setlength{\skip\footins}{0.5cm}    			%脚注与正文的距离           
\renewcommand{\footrulewidth}{0pt}  			%页脚线宽,设为0可以去页脚线

\makeatletter 									%设置双线页眉                                        
\def\headrule{{\if@fancyplain\let\headrulewidth\plainheadrulewidth\fi%
\hrule\@height 1.0pt \@width\headwidth\vskip1pt	%上面线为1pt粗  
\hrule\@height 0.5pt\@width\headwidth  			%下面0.5pt粗            
\vskip-2\headrulewidth\vskip-1pt}      			%两条线的距离1pt        
 \vspace{6mm}}     								%双线与下面正文之间的垂直间距              
\makeatother  

%%%%%%%%%%%%%%%%%%%%%%%%%%%%%%%%%%%%%%%%%%%%%%
\numberwithin{equation}{section}
%\usepackage[thmmarks, amsmath, thref]{ntheorem}
\newtheorem{myThm}{myThm}
\newtheorem*{myDef}{Definition}
\newtheorem*{mySol}{Solution}
\newtheorem*{myProof}{Proof}
\newtheorem*{myRemark}{备注}
\renewcommand{\tilde}{\widetilde}
\renewcommand{\hat}{\widehat}
\newcommand{\indep}{\rotatebox[origin=c]{90}{$\models$}}
\newcommand*\diff{\mathop{}\!\mathrm{d}}

\usepackage{multirow}

%--

%--
\begin{document}

\title{数字信号处理\\
作业一}
\author{你的名字\, 你的学号} 
\maketitle
%%%%%%%% 注意: 使用XeLatex 编译可能会报错,请使用 pdfLaTex 编译 %%%%%%%

\section*{作业提交注意事项}
\begin{tcolorbox}
\begin{enumerate}
  \item[(1)] 本次作业提交截止时间为~\textcolor{red}{\textbf{2022/11/27  23:59:59}},截止时间后不再接收作业,本次作业记零分;
  \item[(2)] 作业提交方式:使用此~LaTex~模板书写解答,只需提交编译生成的~pdf~文件,将~pdf~文件以ftp方式上传,账号为dsp2022,密码为12345asd!@。请远程连接www.lamda.nju.edu.cn,提交到/D:/courses/DSP2022/HW/HW1路径下。
  \item[(3)] 文件命名方式:学号-姓名-作业号-v版本号, 例~ MG1900000-张三-1-v1;如果需要更改已提交的解答,请在截止时间之前提交新版本的解答,并将版本号加一;
  \item[(4)] 未按照要求提交作业,或~pdf~命名方式不正确,将会被扣除部分作业分数。

\end{enumerate}
\end{tcolorbox}


\newpage
\section{[20pts] 信号的周期性}
判断下列信号的周期性,并回答\textbf{是}、\textbf{否}或\textbf{无法判断}。如果是周期信号,请给出其最小正周期。
\begin{enumerate}[(1)]
	\item $x(t)=\sin^2t+\cos\pi t$
	\item $x(t)=(\sin2t+\cos t)^2$
	\item $x(t)=\displaystyle\frac{\cos2t+1+\sin t+\sin2t+\sin3t}{\cos t}$
	\item $x(t)=\sin et+\cos\pi t$
	\item $x(n)=\sin 2kn+\cos 3kn$, $k$为某一正实数。
\end{enumerate}

\begin{framed}
\begin{spacing}{1.5}
    \begin{itemize}
      \item (1)
      
      $\displaystyle x(t) = \sin ^{2}t + \cos \pi t = \frac{1}{2}-\frac{1}{2}\cos 2t + \cos \pi t$
      
      其中 $\cos 2t$ 周期为 $\displaystyle T_1 = \frac{2\pi}{2} = \pi$, $\cos \pi t$ 的周期为 $\displaystyle T_2 = \frac{2\pi}{\pi} = 2$.
      
      由于 $\displaystyle \frac{T_1}{T_2} = \frac{\pi}{2}$ 为无理数, 因此 $x(t)$ 不是周期信号.
      
      \item (2)
      
      $
      \begin{aligned}
      x(t) &= (\sin 2t + \cos t)^{2} \\
      &= \sin^{2} 2t + 2\sin 2t\cos t + \cos^{2} t \\
      &= \frac{1}{2}-\frac{1}{2}\cos 4t + (\sin (2t+t)+\sin (2t-t)) + \frac{1}{2}\cos 2t+\frac{1}{2} \\
      &= -\frac{1}{2}\cos 4t + \sin 3 t + \frac{1}{2}\cos 2t+\sin t+1 \\
      \end{aligned}
      $
      
      由于 $\cos 4t, \sin 3t, \cos 2t, \sin t$ 的周期分别为 $\displaystyle \frac{\pi}{2}, \frac{2\pi}{3}, \pi, 2\pi$.
      
      它们两两间的周期之比为有理数, 因此 $x(t)$ 为周期信号, 周期为它们的最小公倍数 $\displaystyle 2\pi$.
      
      \item (3)
      
      $
      \begin{aligned}
      x(t) &= \frac{\cos 2t + 1 + \sin t + \sin 2t + \sin 3t}{\cos t} \\
      &= \frac{2\cos^{2} t + \sin t + 2\sin t\cos t + 3\sin t - 4\sin^{3} t}{\cos t} \\
      &= \frac{2\cos^{2} t + 4\sin t + 2\sin t\cos t - 4\sin t(1 - \cos^{2} t)}{\cos t} \\
      &= \frac{2\cos^{2} t + 2\sin t\cos t + 4\sin t\cos^{2} t}{\cos t} \\
      &= \frac{2\cos^{2} t + 2\sin t\cos t + 2\sin 2t\cos t}{\cos t} \\
      &= 2\cos t + 2\sin t + 2\sin 2t \\
      \end{aligned}
      $
      
      其中 $\cos t, \sin t, \sin 2t$ 的周期分别为 $2\pi, 2\pi, \pi$.
      
      它们两两间的周期之比为有理数, 因此 $x(t)$ 为周期信号, 周期为它们的最小公倍数 $\displaystyle 2\pi$.
      
      \item (4)
      
      $x(t) = \sin et + \cos \pi t$
      
      其中 $\sin et, \cos \pi t$ 的周期分别为 $\displaystyle \frac{2\pi}{e}, 2$.
      
      由于 $\displaystyle \frac{T_1}{T_2} = \frac{\pi}{e}$ 为无理数, 因此 $x(t)$ 不是周期信号.
      
      \item (5)
      
      $x(n) = \sin 2kn + \cos 3kn$
      
      其中 $\sin 2kn, \cos 3kn$ 的周期分别为 $\displaystyle \frac{\pi}{k}, \frac{2\pi}{3k}$.
      
      由于 $\displaystyle \frac{T_1}{T_2} = \frac{3}{2}$ 为有理数, 因此 $x(t)$ 是周期信号, 周期为最小公倍数 $\displaystyle \frac{2\pi}{k}$.
      
    \end{itemize}
\end{spacing}
\end{framed}


\newpage
\section{[22pts] 连续信号的性质与变换}
已知信号
\begin{equation*}
    \begin{aligned}
    x(t)=\left\{
    \begin{aligned}
    & \frac{1}{2}(t+1), && t\in[-1,1]\\
    & 1, && t\in[1,3]\\
    & 0, && other
    \end{aligned}
    \right.
    \end{aligned}
\end{equation*}
\begin{enumerate}[(1)]
	\item 求$x(t)+x(3-\displaystyle\frac{1}{2}t)u(3-t)$的表达式和图像。
	\item 求$x^{\prime}(t)-x^{\prime\prime}(t)$的表达式和图像。(冲激偶函数用$\delta^{\prime}(t)$表示,其图像为原点向y轴正负半轴分别延伸的箭头)
	\item 设$h(t)=\displaystyle\sum^{\infty}_{n=0}\frac{1}{2^n}\left[x(t+4n)+x(t-4n)\right]$, $h(t)$是否为能量信号或功率信号?请说明理由。
\end{enumerate}
	
\begin{framed}
\begin{spacing}{1.5}
    \begin{itemize}
        \item 你的解答。
    \end{itemize}
\end{spacing}
\end{framed}


\newpage
\section{[28pts] 卷积的计算 }
计算下列各小题的结果:
\begin{enumerate}[(1)]
	\item 设
	\begin{equation*}
        \begin{aligned}
        x(t)=\left\{
        \begin{aligned}
        & 2t+1, && t\in[-1,1]\\
        & 3, && t\in[1,3]\\
        & 0, && other
        \end{aligned}
        \right.
        \end{aligned}
    \end{equation*}
    试求$x(t)*x(t)$的结果。
	\item 求$y(t)=[2e^{-2(t-1)}u(t-2)]*[3e^{-3(t+1)}u(t-1)]$的表达式。
	\item 设$x(n)=\{1,1,4,5,1,4\}$, $y(n)=\{1,9,1,9,8,1\}$, 求$x(n)*y(n)$.
	\item 设
	\begin{equation*}
        \begin{aligned}
        x(n)=\left\{
        \begin{aligned}
        & n, && n=1,2,...k\\
        & 0, && other
        \end{aligned}
        \right.
        \end{aligned}
    \end{equation*}
    试求$x(n)*x(n)$的结果。
\end{enumerate}

\begin{framed}
\begin{spacing}{1.5}
    \begin{itemize}
        \item 你的解答。
    \end{itemize}
\end{spacing}
\end{framed}


\newpage
\section{[30pts] 系统微分方程的求解 }
求解以下微分方程:
\begin{enumerate}[(1)]
	\item $y^{(2)}(t)+7y^{(1)}(t)+12y(t)=2\sin(2t)$ $(t\geqslant0)$, 边界条件$y(0)=0,y^{(1)}(0)=1$.
	\item $y(n)-\displaystyle\frac{3}{4}y(n-1)+\frac{1}{16}y(n-3)=x(n)-x(n-1)$ $(n\geqslant0)$,其中$x(n)=\displaystyle\left(\frac{1}{3}\right)^n$,边界条件$y(0)=y(1)=0,y(2)=1$。
\end{enumerate}

\begin{framed}
\begin{spacing}{1.5}
    \begin{itemize}
        \item 你的解答。
    \end{itemize}
\end{spacing}
\end{framed}


\newpage
\end{document}