% 请确保文件编码为utf-8,使用XeLaTex进行编译,或者通过overleaf进行编译

\documentclass[answers]{exam}  % 使用此行带有作答模块
% \documentclass{exam} % 使用此行只显示题目

\usepackage{xeCJK}
\usepackage{zhnumber}
\usepackage{graphicx}
\usepackage{hyperref}
\usepackage{amsmath}
\usepackage{booktabs}
\usepackage{enumerate}
\usepackage{amssymb}
\usepackage{graphicx} %插入图片的宏包
\usepackage{float} %设置图片浮动位置的宏包
\usepackage{subfigure} %插入多图时用子图显示的宏包


\pagestyle{headandfoot}
\firstpageheadrule
\firstpageheader{南京大学}{机器学习导论}{习题一}
\runningheader{南京大学}
{机器学习导论}
{习题一}
\runningheadrule
\firstpagefooter{}{第\thepage\ 页(共\numpages 页)}{}
\runningfooter{}{第\thepage\ 页(共\numpages 页)}{}


\setlength\linefillheight{.5in}

\renewcommand{\solutiontitle}{\noindent\textbf{解:}\par\noindent}

\renewcommand{\thequestion}{\zhnum{question}}
\renewcommand{\questionlabel}{\thequestion .}
\renewcommand{\thepartno}{\arabic{partno}}
\renewcommand{\partlabel}{\thepartno .}


\begin{document}
\Large
\noindent
% 姓名学号
姓名:方盛俊 \\
学号:201300035 \\
\begin{questions}
  \question [30] \textbf{概率论基础}

  教材附录C介绍了常见的概率分布.
  给定随机变量$X$的概率密度函数如下,

  \begin{equation}
    f_X(x) =
    \begin{cases}
      \frac{1}{4} & 0<x<1;            \\
      \frac{3}{8} & 3<x<5;            \\
      0           & \mbox{otherwise.}
    \end{cases}
  \end{equation}

  \begin{enumerate}
    \item  请计算随机变量$X$的累积分布函数$F_X(x)$;
    \item  随机变量$Y$定义为$Y = 1/X$, 求随机变量$Y$对应的概率密度函数$f_Y(y)$;
    \item  试证明, 对于非负随机变量$Z$, 如下两种计算期望的公式是等价的.
          \begin{equation}
            \label{ch1-eq-expect-1}
            \mathbb{E}[Z] = \int_{z=0}^{\infty}zf(z) \mathrm{d} z.
          \end{equation}

          \begin{equation}
            \label{ch1-eq-expect-2}
            \mathbb{E}[Z] = \int_{z=0}^{\infty}\Pr[Z\geq z] \mathrm{d} z.
          \end{equation}
          同时, 请分别利用上述两种期望公式计算随机变量$X$和$Y$的期望, 验证你的结论.
  \end{enumerate}
  \begin{solution}
    % 请在此处作答
    \begin{enumerate}
      \item 当 $x \le 0$ 时, $F_{X}(x) = 0$

            当 $0 < x \le 1$ 时, $\displaystyle F_{X}(x) = \int_{0}^{x} \frac{1}{4} \mathrm{d}x = \frac{x}{4}$

            当 $1 < x \le 3$ 时, $\displaystyle F_{X}(x) = F_{X}(1) = \frac{1}{4}$

            当 $3 < x \le 5$ 时, $\displaystyle F_{X}(x) = F_{X}(3) + \int_{3}^{x} \frac{3}{8} \mathrm{d}x = \frac{3 x}{8} - \frac{7}{8}$

            当 $x > 5$ 时, $\displaystyle F_{X}(x) = F_{X}(5) = 1$

            综上即有

            \begin{equation}
              F_{X}(x) =
              \begin{cases}
                0,                          & x\le 0      \\
                \frac{x}{4},                & 0 < x \le 1 \\
                \frac{1}{4},                & 1 < x \le 3 \\
                \frac{3x}{8} - \frac{7}{8}, & 3 < x \le 5 \\
                1,                          & x > 5       \\
              \end{cases}
            \end{equation}

      \item 当 $y \le 0$ 时, 有 $x < 0$, 因此 $F_{Y}(x) = 0$

            当 $y > 0$ 时,

            $F_{Y}(y) = \operatorname{Pr}(Y < y) = \operatorname{Pr}(\frac{1}{X} < y) = \operatorname{Pr}(X > \frac{1}{y}) = 1 - F_{X}(\frac{1}{y})$

            因此

            \begin{equation}
              F_{Y}(y) =
              \begin{cases}
                0,                         & y < \frac{1}{5}                 \\
                \frac{15}{8}-\frac{3}{8y}, & \frac{1}{5} \le y < \frac{1}{3} \\
                \frac{3}{4},               & \frac{1}{3} \le y < 1           \\
                1 - \frac{1}{4y},          & y \ge 1                         \\
              \end{cases}
            \end{equation}

            进而有

            \begin{equation}
              f_{Y}(y) =
              \begin{cases}
                0,                 & y < \frac{1}{5}                 \\
                \frac{3}{8 y^{2}}, & \frac{1}{5} \le y < \frac{1}{3} \\
                0,                 & \frac{1}{3} \le y < 1           \\
                \frac{1}{4 y^{2}}, & y \ge 1                         \\
              \end{cases}
            \end{equation}

      \item 首先观察到 $\displaystyle Z = \int_{0}^{Z} 1 \mathrm{d}t = \int_{0}^{\infty} \mathbb{I}(Z > t) \mathrm{d}t$

            与公式 $\displaystyle \mathbb{E}[Z] = \int_{0}^{\infty} zf(z) \mathrm{d}z$

            则有

            $
              \begin{aligned}
                \mathbb{E}[Z] & = \displaystyle \mathbb{E}[\int_{0}^{\infty} \mathbb{I}(Z > t) \mathrm{d}t]                                                                          \\
                              & = \displaystyle \int_{0}^{\infty} f(z)\int_{0}^{\infty} \mathbb{I}(z > t) \mathrm{d}t \mathrm{d}z                                                    \\
                              & = \displaystyle \int_{0}^{\infty} [\int_{0}^{\infty} \mathbb{I}(z > t)f(z) \mathrm{d}z] \mathrm{d}t                                                  \\
                              & = \displaystyle \int_{0}^{\infty} [\int_{0}^{t} \mathbb{I}(z > t)f(z) \mathrm{d}z + \int_{t}^{\infty} \mathbb{I}(z > t)f(z) \mathrm{d}z] \mathrm{d}t \\
                              & = \displaystyle \int_{0}^{\infty} [\int_{t}^{\infty} f(z) \mathrm{d}z] \mathrm{d}t                                                                   \\
                              & = \int_{0}^{\infty} \operatorname{Pr}[Z \ge t] \mathrm{d}t                                                                                           \\
                              & = \int_{0}^{\infty} \operatorname{Pr}[Z \ge z] \mathrm{d}z                                                                                           \\
              \end{aligned}
            $

            计算随机变量 $X$ 的期望:

            $\displaystyle \mathbb{E}[X] = \int_{0}^{\infty} xf_{X}(x) \mathrm{d}x = \int_{0}^{1} \frac{1}{4} x\mathrm{d}x + \int_{3}^{5} \frac{3}{8} x\mathrm{d}x = \frac{25}{8}$

            $
              \begin{aligned}
                \mathbb{E}[X] & = \int_{0}^{\infty} \operatorname{Pr}[X \ge x] \mathrm{d}x                                                \\
                              & = \int_{0}^{1} (1 - \frac{x}{4}) \mathrm{d}x + \int_{1}^{3} (1 - \frac{1}{4}) \mathrm{d}x                 \\
                              & \quad + \int_{3}^{5} (1 - \frac{3x}{8} + \frac{7}{8}) \mathrm{d}x + \int_{5}^{\infty} (1 - 1) \mathrm{d}x \\
                              & = \frac{7}{8} + \frac{3}{2} + \frac{3}{4}                                                                 \\
                              & = \frac{25}{8}                                                                                            \\
                \\
              \end{aligned}
            $

            计算随机变量 $Y$ 的期望:

            $\displaystyle \mathbb{E}[Y] = \int_{0}^{\infty} yf_{Y}(y) \mathrm{d}y = \int_{\frac{1}{5}}^{\frac{1}{3}} \frac{3}{8y^{2}} \cdot y \mathrm{d}y + \int_{1}^{\infty} \frac{1}{4y^{2}} \cdot y \mathrm{d}y = - \frac{3 \ln{\left(3 \right)}}{8} + \frac{3 \ln{\left(5 \right)}}{8} + \infty$

            $
              \begin{aligned}
                \mathbb{E}[Y] & = \int_{\frac{1}{5}}^{\frac{1}{3}} (1 - \frac{15}{8} + \frac{3}{8y}) \mathrm{d}y + \int_{\frac{1}{3}}^{1} (1 - \frac{3}{4}) \mathrm{d}y + \int_{1}^{\infty} \frac{1}{4y} \mathrm{d}y \\
                              & = - \frac{3 \ln{\left(3 \right)}}{8} - \frac{7}{60} + \frac{3 \ln{\left(5 \right)}}{8} + \frac{1}{6} + \infty                                                                        \\
              \end{aligned}
            $

            均为不收敛.
    \end{enumerate}
  \end{solution}


  \question [40] \textbf{评估方法}

  教材2.2.3节描述了自助法~(bootstrapping), 下面考虑将自助法用于对统计量估计这一场景, 并对自助法做进一步分析.
  考虑$m$个从分布$p(x)$中独立同分布抽取的(互不相等的)观测值$x_1, x_2, \ldots, x_m$, $p(x)$的均值为$\mu$, 方差为$\sigma^2$. 通过$m$个样本, 可使用如下方式估计分布的均值
  \begin{equation}
    \bar{x}_m = \frac{1}{m} \sum_{i=1}^{m} x_{i}\;,\label{ch2_eq:estimate_mean}
  \end{equation}
  和方差
  \begin{equation}
    \bar{\sigma}^2_m=\frac{1}{m-1} \sum_{i=1}^{m}\left(x_{i}-\bar{x}_m\right)^{2}\label{ch2_eq:estimate_variance}
  \end{equation}
  设$x^*_1, x^*_2, \ldots, x^*_m$为通过自助法采样得到的结果, 且
  \begin{equation}
    \bar{x}^*_m = \frac{1}{m} \sum_{i=1}^{m} x^*_{i}\;,
  \end{equation}
  \begin{enumerate}
    \item 请证明$\mathbb E[\bar{x}_m] = \mu$且$\mathbb E[\bar{\sigma}^2_m] = \sigma^2$;
    \item 计算$var[\bar{x}_m]$;
    \item 计算$\mathbb E[\bar{x}^*_m \mid x_1, \ldots, x_m]$和$var[\bar{x}^*_m \mid x_1, \ldots, x_m]$;
    \item 计算$\mathbb E[\bar{x}^*_m]$和$var[\bar{x}^*_m]$;
    \item 针对上述证明分析自助法和交叉验证法的不同.
  \end{enumerate}

  \begin{solution}
    %请在此处作答
    \begin{enumerate}
      \item 对于期望有

            \begin{equation}
              \mathbb{E}[\bar{x}_m] = \mathbb{E}[\frac{1}{m}\sum_{i=1}^{m} x_i] = \frac{1}{m} \sum_{i=1}^{m} \mathbb{E}[x_i] = \frac{1}{m} \sum_{i=1}^{m} \mu = \mu
            \end{equation}

            对于第二个式子有

            \begin{equation}
              \begin{aligned}
                \mathbb{E}[\bar{\sigma}_m^{2}] & = \frac{1}{m - 1} \mathbb{E}[\sum_{i=1}^{m} (x_i - \bar{x}_m)^{2}]                                                                            \\
                                               & = \frac{1}{m - 1} \mathbb{E}[\sum_{i=1}^{m} x_{i}^{2} - 2 \bar{x}_{m} \sum_{i=1}^{m} x_{i} + \sum_{i=1}^{m}\bar{x}_{m}^{2}]                   \\
                                               & = \frac{1}{m - 1} \mathbb{E}[\sum_{i=1}^{m} x_{i}^{2} - m \bar{x}_{m}^{2}]                                                                    \\
                                               & = \frac{1}{m - 1} (\sum_{i=1}^{m}\mathbb{E}[x_{i}^{2}] - m \mathbb{E}[\bar{x}_{m}^{2}])                                                       \\
                                               & = \frac{1}{m - 1} (\sum_{i=1}^{m}(\mathbb{E}[x_{i}^{2}] - \mathbb{E}[x_i]^{2}) - m (\mathbb{E}[\bar{x}_{m}^{2}] - \mathbb{E}[\bar{x}_m]^{2})) \\
                                               & = \frac{1}{m - 1} (\sum_{i=1}^{m}\operatorname{Var}[x_i] - m \operatorname{Var}[\bar{x}_m])                                                   \\
                                               & = \frac{1}{m - 1} (m\cdot \sigma^{2} - m \cdot (\frac{1}{m^{2}} \cdot m \sigma^{2}))                                                          \\
                                               & = \sigma^{2}                                                                                                                                  \\
              \end{aligned}
            \end{equation}

      \item 计算方差得

            \begin{equation}
              \operatorname{Var}[\bar{x}_m] = \frac{1}{m^{2}} \cdot m \sigma^{2} = \frac{1}{m}\sigma^{2}
            \end{equation}

      \item 对于任意一个自助法得到的样本 $x_i^{*}$ 有

            \begin{equation}
              \mathbb{E}[x_i^{*} | x_1, \cdots, x_m] = \frac{1}{m}\sum_{i=1}^{m} x_i = \bar{x}_m
            \end{equation}

            \begin{equation}
              \begin{aligned}
                \operatorname{Var}[x_i^{*} | x_1, \cdots, x_m]
                 & = \mathbb{E}[(x_i^{*} - \mathbb{E}[x_i^{*} | x_1, \cdots, x_m])^{2} | x_1, \cdots, x_m] \\
                 & = \mathbb{E}[(x_i^{*} - \bar{x}_m)^{2} | x_1, \cdots, x_m]                              \\
                 & = \frac{1}{m}\sum_{i=1}^{m} (x_i - \bar{x}_m)^{2}                                   \\
                 & = \frac{m-1}{m}\bar{\sigma}_m^{2}                                                   \\
              \end{aligned}
            \end{equation}

            因此

            \begin{equation}
              \mathbb{E}[\bar{x}_m^{*} | x_1, \cdots, x_m] = \frac{1}{m}\sum_{i=1}^{m} \mathbb{E}[x_i^{*} | x_1, \cdots, x_m] = \bar{x}_m
            \end{equation}

            \begin{equation}
              \operatorname{Var}[\bar{x}_m^{*} | x_1, \cdots, x_m] = \frac{1}{m^{2}} \cdot \sum_{i=1}^{m} \operatorname{Var}[x_i^{*} | x_1, \cdots, x_m] = \frac{m-1}{m^{2}}\bar{\sigma}_m^{2}
            \end{equation}

      \item 对于任意一个自助法得到的样本 $x_i^{*}$ 有期望

            \begin{equation}
              \mathbb{E}[x_i^{*}] = \sum_{i=1}^{m} \frac{1}{m}\mathbb{E}[x_i] = \mu
            \end{equation}

            和方差

            \begin{equation}
              \begin{aligned}
                \operatorname{Var}[x_i^{*}] & = \mathbb{E}[x_i^{*2}] - \mathbb{E}[x_i^{*}]^{2}                        \\
                                            & = \sum_{i=1}^{m} \frac{1}{m}\mathbb{E}[x_i^{2}] - \mu^{2}               \\
                                            & = \frac{1}{m}\sum_{i=1}^{m} (\mathbb{E}[x_i^{2}] - \mathbb{E}[x_i]^{2}) \\
                                            & = \frac{1}{m}\sum_{i=1}^{m} \operatorname{Var}[x_i]                     \\
                                            & = \sigma^{2}
              \end{aligned}
            \end{equation}

            因此

            \begin{equation}
              \mathbb{E}[\bar{x}_m^{*}] = \frac{1}{m}\sum_{i=1}^{m} \mathbb{E}[x_i^{*}] = \mu
            \end{equation}

            \begin{equation}
              \operatorname{Var}[\bar{x}_m^{*}] = \frac{1}{m^{2}} \cdot \sum_{i=1}^{m} \operatorname{Var}[x_i^{*}] = \frac{1}{m}\sigma^{2}
            \end{equation}

      \item 交叉验证法, 特别是其中的留一法, 同一个样例不会被抽取多次, 和实际上用 $D$ 训练出的模型较为相似, 所以在大数据集上相对准确, 但是计算开销较大.

            自助法, 虽然期望和方差仍然维持与原数据集相同, 但是会重复抽取相同的样本, 改变了初始数据集的分布, 会引入估计偏差, 一般用于数据集较小的时候.
    \end{enumerate}
  \end{solution}


  \question [30] \textbf{性能度量}

  教材2.3节介绍了机器学习中常用的性能度量. 假设数据集包含8个样例, 其对应的真实标记和学习器的输出值(从大到小排列)如表~\ref{table:roc}所示. 该任务是一个二分类任务, 标记1和0表示真实标记为正例或负例.
  学习器的输出值代表学习器认为该样例是正例的概率.
  \begin{table}[!h]
    \centering
    \caption{样例表} \vspace{2mm}\label{table:roc}
    \begin{tabular}{c|c c c c c c c c c c c}\toprule
      样例         & $x_1$ & $x_2$ & $x_3$ & $x_4$ & $x_5$ & $x_{6}$ & $x_{7}$ & $x_{8}$ \\
      \midrule
      标记         & 1     & 1     & 0     & 1     & 0     & 1       & 0       & 0       \\
      \midrule
      分类器输出值 & 0.81  & 0.74  & 0.62  & 0.55  & 0.44  & 0.35    & 0.25    & 0.21    \\
      \bottomrule
    \end{tabular}
  \end{table}
  \begin{enumerate}
    \item 计算P-R曲线每一个端点的坐标并绘图;
    \item 计算ROC曲线每一个端点的坐标并绘图, 计算AUC;
  \end{enumerate}
  \begin{solution}
    \begin{enumerate}
      \item 计算得

            \begin{table}[H]
              \centering
              \caption{P-R 曲线} \vspace{2mm}\label{table:roc}
              \begin{tabular}{c|c c c c c c c c c c c}\toprule
                点 & 1   & 2    & 3   & 4    & 5    & 6    & 7    & 8    & 9   \\
                \midrule
                P  & 1.0 & 1.0  & 1.0 & 0.67 & 0.75 & 0.6  & 0.67 & 0.57 & 0.5 \\
                \midrule
                R  & 0.0 & 0.25 & 0.5 & 0.5  & 0.75 & 0.75 & 1.0  & 1.0  & 1.0 \\
                \bottomrule
              \end{tabular}
            \end{table}

            \begin{figure}[H]
              \centering
              \includegraphics[width=0.9\textwidth]{Figure_1.png}
              \caption{P-R 曲线}
              \label{Fig.main1}
            \end{figure}

      \item 计算得

            \begin{table}[H]
              \centering
              \caption{P-R 曲线} \vspace{2mm}\label{table:roc}
              \begin{tabular}{c|c c c c c c c c c c c}\toprule
                点  & 1   & 2    & 3   & 4    & 5    & 6    & 7   & 8    & 9   \\
                \midrule
                TPR & 0.0 & 0.25 & 0.5 & 0.5  & 0.75 & 0.75 & 1.0 & 1.0  & 1.0 \\
                \midrule
                FPR & 0.0 & 0.0  & 0.0 & 0.25 & 0.25 & 0.5  & 0.5 & 0.75 & 1.0 \\
                \bottomrule
              \end{tabular}
            \end{table}

            \begin{figure}[H]
              \centering
              \includegraphics[width=0.9\textwidth]{Figure_2.png}
              \caption{ROC 曲线}
              \label{Fig.main1}
            \end{figure}

            AUC 面积为 $S = \frac{1}{2}\sum_{i=1}^{m-1} (x_{i+1}-x_{i})\cdot (y_{i} + y_{i+1}) = 0.8125$
    \end{enumerate}
  \end{solution}

\end{questions}


\end{document}