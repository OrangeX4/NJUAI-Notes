\documentclass[UTF8]{ctexart}
% 插入图片所使用的宏包
\usepackage{graphicx}
% 固定图片
\usepackage{float}
% 超链接和目录颜色
\usepackage[colorlinks, urlcolor=blue, linkcolor=black]{hyperref}
% 用于生成随机中文
\usepackage{zhlipsum}
% 设置section左对齐,默认是居中
% \CTEXsetup[format={\Large\bfseries}]{section}
\ctexset{ section = { format=\Large\bfseries } }
% 设置章节编号
\renewcommand\thesection{\chinese{section}、}
\renewcommand\thesubsection{\arabic{subsection}.}
\renewcommand\thesubsubsection{\alph{subsubsection})}
% 设置章节编号与章节标题的距离
\makeatletter
\renewcommand\@seccntformat[1]{%
    {\csname the#1\endcsname}\hspace{0.1em}
}
\makeatother

% 在这里设置你的标题和姓名
\title{实验一: 双控开关和表决器}
\author{方盛俊}

\begin{document}

% 标题页
\begin{titlepage}
    \newcommand{\HRule}{\rule{\linewidth}{0.5mm}}
    \center
    \quad \\[1.5cm]
    \textbf{\Huge 数字电路与数字系统实验报告 } \\[3cm]
    \makeatletter
    \HRule \\[0.8cm]
    { \huge \bfseries \@title} \\[0.4cm]
    \HRule \\[2cm]
    \begin{tabular}{rl}
        \textbf{\Large 院系:} & \textbf{\Large 人工智能学院}               \\[0.5cm]
        \textbf{\Large 姓名:} & \textbf{\Large \@author}                   \\[0.5cm]
        % 设定你的学号, 班级和邮箱
        \textbf{\Large 学号:} & \textbf{\Large 201300035}                  \\[0.5cm]
        \textbf{\Large 班级:} & \textbf{\Large 人工智能 20 级 2 班}        \\[0.5cm]
        \textbf{\Large 邮箱:} & \textbf{\Large 201300035@smail.nju.edu.cn} \\[0.5cm]
        % 默认为今天, 如果要修改为你需要的时间, 请将 \today 替换为你需要的时间
        \textbf{\Large 时间:} & \textbf{\Large \today}                     \\[2cm]
    \end{tabular}
    \vfill
\end{titlepage}

% 生成目录
\newpage
\tableofcontents
\newpage

% 加大段落间距
\setlength{\parskip}{1em}

% 主体内容
% 第一部分: 实验目的
\section{实验目的}
% 列表语法
\begin{enumerate}
    \item 熟悉 Logisim 软件的基本使用方法
    \item 掌握使用晶体管实现基本逻辑部件的方法
    \item 利用基础元器件库设计简单数字电路
    \item 了解子电路的设计和应用
    \item 掌握分线器、隧道、探针等组件的使用方法
\end{enumerate}


% 第二部分: 实验目的
\section{实验原理(背景知识)}

\subsection{晶体管}

我们电路设计中使用的晶体管,即基于金属氧化物半导体场效应晶体管的 \textbf{CMOS}(Complementary
Metal-Oxide Semiconductor)。

MOS晶体管三极:

\begin{itemize}
    \item 栅极 gate
    \item 源极 source
    \item 漏极 drain
\end{itemize}

MOS晶体管可被模型化为一种3端子压控电阻导体,将电压加到一
个端子,来控制其他两个端子间的电阻。

MOS晶体管分为:
\begin{itemize}
    \item N 沟道型 NMOS,N型杂质有磷或者锑
    \item P 沟道型 PMOS,P型杂质有硼或者铟
\end{itemize}

栅极和源极之间电压 $V_{gs}$ 控制源极和漏极间电阻 $R_{ds}$ 的大小,进而可以组合出不同的基本电路器件。

\subsection{逻辑表达式}

在实际构建组合电路之前,我们常常先把真值表写出,例如异或门:

\begin{center}
    \centering
    \begin{tabular}{cc|c}
        \hline
        A & B & F \\
        \hline
        0 & 0 & 0 \\
        0 & 1 & 1 \\
        1 & 0 & 1 \\
        1 & 1 & 0 \\
        \hline
    \end{tabular}
\end{center}

进而根据真值表推导出规范的逻辑表达式,常常是最简化的\textbf{积之和表达式},如异或的积之和表达式:

\[
    F = \overline{A}B + A\overline{B}
\]

进而根据推导出来的逻辑表达式,选取需要的基本逻辑门,构建该电路。

\subsection{Logisim 的基础使用}

\textbf{Logisim} 是 Carl Burch 开发的一个设计和仿真数字逻辑电路的图形化教学工具。非常便于学习数字逻辑电路设计的基本概念,能够从简单的子电路分层构建较复杂的数字电路。

2001年4月-2011年5月,版本V2.7.1,基于Java开发,跨平台运行,按照GNU开源协议授权使用。Windows平台使用EXE文件,MAC或Linux平台使用JAR文件。

在SourceForge上有多个Logisim项目,其中意大利语版本2017年开始修改,在Java 10及以上版本上重新编译,修改bug,添加新组件。最新版本: V2.16.1.0 2020-04-18

电路文件类型为 circ,格式为 xml,所有子电路都是以这样的标记出现,<circuit name="xxx"></circuit>,包含电路外观。

\begin{figure}[H]
    \centering
    \includegraphics[width = \textwidth]{images/logisim.png}
\end{figure}

在查阅相关文档之后,发现有一个命令可以模仿头歌平台的运行,本地进行样例测试。

即命令:

\begin{center}
    \textbf{java -jar logisim.jar file.circ -tty table}
\end{center}

使用这个命令,就可以较为方便地先在本地进行样例测试,以防提交过程中过多的错误尝试。


% 第三部分: 实验目的
\section{实验环境/器材}

\paragraph{Logisim-ITA V2.16.1.2}~{}
\newline
\href{https://sourceforge.net/projects/logisimit/}{https://sourceforge.net/projects/logisimit/}

\paragraph{头歌线上评测平台}~{}
\newline
\href{https://www.educoder.net/classrooms/10924/}{https://www.educoder.net/classrooms/10924/}


% 第四部分: 实验内容
\section{实验内容}

\subsection{利用晶体管构建或门}

\subsubsection{实验原理}

二极管、三极管、MOS 管都是基础的电器元件,其中 MOS 管(即晶体管)是现代
数字电路的基础。

与门、或门、非门等门电路是数字电路中常用的基础门电路。本次实验的
内容是通过利用晶体管构建实现基本逻辑运算的门电路,这里选择的是或门电路。

根据数字电路原理,或门是由或非门级联一个反相器构成。即公式:

\[
    OR(x, y) = NOT(NOR(x, y))
\]

或非门和反向器的原理如图:

\begin{figure}[H]
    \centering
    \includegraphics[width = .8\textwidth]{images/gates.jpg}
\end{figure}


\subsubsection{实验步骤}


\textbf{1) 构建非门}

需要电路部件:1 对晶体管、1 个输入引脚、1 个输出引脚、1 个电源、1 个接地线。

放入一个晶体管 P-Type,朝向为 South;再放入一个晶体管 N-Type,朝向为 North。

按照原理图,将非门的电路连接好,如图:

\begin{figure}[H]
    \centering
    \includegraphics[width = .6\textwidth]{images/not.jpg}
\end{figure}


\textbf{2) 构建或非门}

需要电路部件:2 对晶体管、2 个输入引脚、1 个输出引脚、1 个电源、1 个接地线。

放入两个晶体管 P-Type,朝向为 South;再放入两个晶体管 N-Type,朝向为 North。

按照原理图,将或非门的电路连接好,如图:

\begin{figure}[H]
    \centering
    \includegraphics[width = .6\textwidth]{images/nor.jpg}
\end{figure}

\textbf{3) 利用非门和或非门构建或门}

根据公式

\[
    OR(x, y) = NOT(NOR(x, y))
\]

将非门和或非门连接:

\begin{figure}[H]
    \centering
    \includegraphics[width = .6\textwidth]{images/or.jpg}
\end{figure}

一个或门电路就搭好了。


\subsubsection{仿真验证}

我们先在本地 Logisim 软件进仿真验证,按下快捷键 \textbf{Ctrl + K}, Logisim 便开始了仿真。

\begin{figure}[H]
    \centering
    \includegraphics[width = \textwidth]{images/sim.jpg}
\end{figure}

可见,电路的输出符合描述。

再在本地运行命令:

\begin{center}
    \textbf{java -jar logisim.jar exp1/exp1-1.circ -tty table}
\end{center}

便可得到电路模拟输出:

\begin{figure}[H]
    \centering
    \includegraphics[width = \textwidth]{images/native.jpg}
\end{figure}

也符合预期,说明电路在本地仿真验证通过了。



\subsubsection{实验结果}

提交到头歌平台,获得输出如下:

\begin{figure}[H]
    \centering
    \includegraphics[width = 0.8\textwidth]{images/test.jpg}
\end{figure}

实验结果一致。


\subsection{双控开关}

\subsubsection{实验原理}

双控开关是生活中常见的开关,双控开关是指两个拨动开关能够同时控制一盏灯。

例如,在卧室中可以通过门口或床头的开关来开关卧室的灯,这样开关灯时就不用跑来跑去了。

实际上,双控开关的电路原理就是异或电路,两个开关的状态作为输入,灯亮与否作为输出的异或电路。

根据数字电路原理,异或电路可以由一个公式表示:

\[
    F = A \oplus B = \overline{A}B + A\overline{B}
\]


\subsubsection{实验步骤}


\textbf{1) 放入电路原件}

分析公式可知,我们需要电路部件:2 个输入引脚,1 个输出引脚,2 个非门,2 个 2 输入与门、1 个 2 输入或门。

\textbf{2) 连接异或电路}

按照公式,两个输入分别取反之后作为两个与门的输入,两个与门的输出再作为或门的输入,最后将或门输出导向输出引脚,如图:

\begin{figure}[H]
    \centering
    \includegraphics[width = .6\textwidth]{images/xor.jpg}
\end{figure}



\subsubsection{仿真验证}

我们先在本地 Logisim 软件进仿真验证,按下快捷键 \textbf{Ctrl + K}, Logisim 便开始了仿真。

\begin{figure}[H]
    \centering
    \includegraphics[width = \textwidth]{images/sim2.jpg}
\end{figure}

可见,电路的输出符合描述。

再在本地运行命令:

\begin{center}
    \textbf{java -jar logisim.jar exp1/exp1-2.circ -tty table}
\end{center}

便可得到电路模拟输出:

\begin{figure}[H]
    \centering
    \includegraphics[width = \textwidth]{images/native2.jpg}
\end{figure}

也符合预期,说明电路在本地仿真验证通过了。



\subsubsection{实验结果}

提交到头歌平台,获得输出如下:

\begin{figure}[H]
    \centering
    \includegraphics[width = 0.8\textwidth]{images/test2.jpg}
\end{figure}

实验结果一致。


\subsection{多数表决器}

\subsubsection{实验原理}

三路多数表决器,顾名思义,就是三个人进行投票,两票及以上为同意,零票或一票为不同意。

用电路来描述,就是有两路及以上输入为 1 时,输出才为 1,否则为 0,用真值表描述如下:

\begin{center}
    \centering
    \begin{tabular}{ccc|c}
        \hline
        A & B & C & F \\
        \hline
        0 & 0 & 0 & 0 \\
        0 & 0 & 1 & 0 \\
        0 & 1 & 0 & 0 \\
        0 & 1 & 1 & 1 \\
        1 & 0 & 0 & 0 \\
        1 & 0 & 1 & 1 \\
        1 & 1 & 0 & 1 \\
        1 & 1 & 1 & 1 \\
        \hline
    \end{tabular}
\end{center}

用公式来表达也很简单:

\[
    F = A \cdot B + B \cdot C + A \cdot C
\]

\subsubsection{实验步骤}


\textbf{1) 放入电路原件}

分析公式可知,我们需要电路部件:3 个输入引脚,1 个输出引脚,3 个 2 输入与门、1 个 3 输入或门。

\textbf{2) 连接异或电路}

连接电路如图:

\begin{figure}[H]
    \centering
    \includegraphics[width = .8\textwidth]{images/vote.jpg}
\end{figure}


\subsubsection{仿真验证}

在本地运行命令

\begin{center}
    \textbf{java -jar logisim.jar exp1/exp1-3.circ -tty table}
\end{center}

便可得到电路模拟输出:

\begin{figure}[H]
    \centering
    \includegraphics[width = \textwidth]{images/native3.jpg}
\end{figure}

也符合预期,说明电路在本地仿真验证通过了。

\subsubsection{实验结果}

提交到头歌平台,获得输出如下:

\begin{figure}[H]
    \centering
    \includegraphics[width = 0.8\textwidth]{images/test3.jpg}
\end{figure}

实验结果一致。


% 第五部分: 实验中遇到的问题和解决的办法
\section{实验中遇到的问题和解决的办法}

\subsection{本地仿真模拟与平台上运行不一致}

在本地运行相关的电路代码的时候,有时候会碰到本地和平台上输出结果不一致。

如多数表决器,在本地运行的时候输出如下:

\begin{figure}[H]
    \centering
    \includegraphics[width = \textwidth]{images/native0.png}
\end{figure}

而在平台上运行时却成了这样:

\begin{figure}[H]
    \centering
    \includegraphics[width = \textwidth]{images/test0.png}
\end{figure}

说明平台上的运行环境可能和本地运行有一定差异。

这个问题难以独自解决,于是请教了老师,最后通过修改平台的评判代码的方式,解决了这个问题。


% 第六部分: 实验得到的启示/意见与建议
% \section{实验得到的启示/意见与建议}


\end{document}