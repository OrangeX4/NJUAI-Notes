\documentclass[a4paper,UTF8]{article}
\usepackage{ctex}
\usepackage[margin=1.25in]{geometry}
\usepackage{color}
\usepackage{graphicx}
\usepackage{amssymb}
\usepackage{amsmath}
\usepackage{amsthm}
\usepackage{enumerate}
\usepackage{bm}
\usepackage{hyperref}
\usepackage{pgfplots}
\usepackage{epsfig}
\usepackage{color}
\usepackage{tcolorbox}
\usepackage{mdframed}
\usepackage{lipsum}
\usepackage{framed}
\usepackage{setspace}
\usepackage{listings}

\newmdtheoremenv{thm-box}{myThm}
\newmdtheoremenv{prop-box}{Proposition}
\newmdtheoremenv{def-box}{定义}

\setlength{\evensidemargin}{.25in}
\setlength{\textwidth}{6in}
\setlength{\topmargin}{-0.5in}
\setlength{\topmargin}{-0.5in}
% \setlength{\textheight}{9.5in}
%%%%%%%%%%%%%%%%%%此处用于设置页眉页脚%%%%%%%%%%%%%%%%%%
\usepackage{fancyhdr}                                
\usepackage{lastpage}                                           
\usepackage{layout}                                             
\footskip = 10pt 
\pagestyle{fancy}                    % 设置页眉                 
\lhead{2022年秋季}                    
\chead{数字信号处理}                                                
% \rhead{第\thepage/\pageref{LastPage}页} 
\rhead{作业三}                                                                                               
\cfoot{\thepage}                                                
\renewcommand{\headrulewidth}{1pt}  			%页眉线宽,设为0可以去页眉线
\setlength{\skip\footins}{0.5cm}    			%脚注与正文的距离           
\renewcommand{\footrulewidth}{0pt}  			%页脚线宽,设为0可以去页脚线

\makeatletter 									%设置双线页眉                                        
\def\headrule{{\if@fancyplain\let\headrulewidth\plainheadrulewidth\fi%
		\hrule\@height 1.0pt \@width\headwidth\vskip1pt	%上面线为1pt粗  
		\hrule\@height 0.5pt\@width\headwidth  			%下面0.5pt粗            
		\vskip-2\headrulewidth\vskip-1pt}      			%两条线的距离1pt        
	\vspace{6mm}}     								%双线与下面正文之间的垂直间距              
\makeatother  

%%%%%%%%%%%%%%%%%%%%%%%%%%%%%%%%%%%%%%%%%%%%%%
\numberwithin{equation}{section}
%\usepackage[thmmarks, amsmath, thref]{ntheorem}
\newtheorem{myThm}{myThm}
\newtheorem*{myDef}{Definition}
\newtheorem*{mySol}{Solution}
\newtheorem*{myProof}{Proof}
\newtheorem*{myRemark}{备注}
\renewcommand{\tilde}{\widetilde}
\renewcommand{\hat}{\widehat}
\newcommand{\indep}{\rotatebox[origin=c]{90}{$\models$}}
\newcommand*\diff{\mathop{}\!{d}}
\setlength{\parindent}{0pt}
\usepackage{multirow}


\lstset{language=Matlab}%这条命令可以让LaTeX排版时将Matlab关键字突出显示
\lstset{
	breaklines,%这条命令可以让LaTeX自动将长的代码行换行排版
	basicstyle=\footnotesize\ttfamily, % Standardschrift
	backgroundcolor=\color[rgb]{0.95,0.95,0.95},
	keywordstyle=\color{blue},
	commentstyle=\color{cyan},
	tabsize=4,numbers=left,
	numberstyle=\tiny,
	frame=single,
	%numbers=left, % Ort der Zeilennummern
	numberstyle=\tiny, % Stil der Zeilennummern
	%stepnumber=2, % Abstand zwischen den Zeilennummern
	numbersep=5pt, % Abstand der Nummern zum Text
	tabsize=2, % Groesse von Tabs
	extendedchars=false, %
	breaklines=true, % Zeilen werden Umgebrochen
	keywordstyle=\color{red},%这一条命令可以解决代码跨页时, 章节标题, 页眉等汉字不显示的问题
	stringstyle=\color{white}\ttfamily, % Farbe der String
	showspaces=false, % Leerzeichen anzeigen ?
	showtabs=false, % Tabs anzeigen ?
	xleftmargin=17pt,
	framexleftmargin=17pt,
	framexrightmargin=5pt,
	framexbottommargin=4pt,
	%backgroundcolor=\color{lightgray},
	showstringspaces=false % Leerzeichen in Strings anzeigen ?
}
\renewcommand{\lstlistingname}{CODE}
\lstloadlanguages{% Check Dokumentation for further languages ...
	%[Visual]Basic
	%Pascal
	%C
	Python
	%XML
	%HTML
	%Java
}

%--

%--
\begin{document}
	
	\title{数字信号处理\\
		作业三}
	\author{方盛俊\, 201300035} 
	\maketitle
	%%%%%%%% 注意: 使用XeLatex 编译可能会报错,请使用 pdfLaTex 编译 %%%%%%%
	
	\section*{作业提交注意事项}
	\begin{tcolorbox}
		\begin{enumerate}
			\item[(1)] 本次作业提交截止时间为~\textcolor{red}{\textbf{2022/12/07  23:59:59}},截止时间后不再接收作业,本次作业记零分;
			\item[(2)] 作业提交方式:使用此 \LaTeX 模板书写解答,只需提交编译生成的~pdf~文件,将~pdf~文件以 sftp 方式上传,账号为 dsp2022,密码为 12345asd!@。请远程连接 sftp://www.lamda.nju.edu.cn,提交到 /D:/courses/DSP2022/HW/HW3 路径下。
			\item[(3)] 文件命名方式:学号-姓名-作业号-v版本号, 例~ MG1900000-张三-3-v1;如果需要更改已提交的解答,请在截止时间之前提交新版本的解答,并将版本号加一;
			\item[(4)] 未按照要求提交作业,或~pdf~命名方式不正确,将会被扣除部分作业分数。
			
		\end{enumerate}
	\end{tcolorbox}
	
	
	\newpage
	\section{[30pts] 信号的抽样}
	\begin{itemize}
		\item[1.]有一理想抽样系统, 抽样频率为 $\Omega_{s}=6 \pi$, 抽样后经理想低通滤波器 $H_a({j} \Omega)$ 还原,其中
		$$
		H_a({j} \Omega)= \begin{cases}\frac{1}{2}, & |\Omega|<3 \pi \\ 0, & |\Omega| \geqslant 3 \pi\end{cases}
		$$
		
		现有两个输入 $x_{a_1}(t)=\cos 2 \pi t$, $x_{a_2}(t)=\cos 5 \pi t$. 问输出信号 $y_{a_1}(t), y_{a_2}(t)$ 有无失真? 为什么?
		\item[2.] 已知实信号 $x(t)$ 的奈奎斯特频率为 $\omega_0$, 试计算对下列各信号抽样不混叠的最小抽样频率.
		\begin{itemize}
			\item[(1)]$ x(t)+x(t-t_0) $
			\item[(2)]$ x'(t) $
			\item[(3)]$ x^2(t) $
			\item[(4)]$ x(t)\cos \omega_0 t $
		\end{itemize}
	\end{itemize}
	
	\begin{framed}
		\begin{spacing}{1.5}
			\begin{itemize}
        \item 1.

        (1)
        
        由于 $x_{a_1} = \cos 2\pi t$ 的傅里叶变换为 $X_{a_1} = \pi \delta(\Omega - 2\pi) + \pi \delta(\Omega + 2\pi)$
        
        即有 $x_{a_1} = \cos 2\pi t$ 的最高频率为 $\Omega_m = 2\pi$
        
        由 $\Omega_s = 6\pi > 2 \Omega_m = 4\pi$ 与采样定理可知没有发生混叠, 选用适当的低通滤波器可以恢复.
        
        因此抽样后傅里叶变换为 $\displaystyle \frac{\Omega_s}{2\pi}\sum_{n=-\infty}^{+\infty}X_{a_1}(j(\Omega-n \Omega_s)) = 3\sum_{n=-\infty}^{+\infty}X_{a_1}(j(\Omega-n \Omega_s))$
        
        经过低通滤波器 $\displaystyle H_a(j\Omega) = \begin{cases}
            \frac{1}{2}, & |\Omega| < 3\pi \\
            0, & |\Omega| \ge 3\pi \\
        \end{cases}$
        
        可得 $\displaystyle Y_{a_1}(j\Omega) = \frac{3}{2}X_{a_1}(j\Omega)$, 在相位谱上没有发生失真, 但是在幅度谱上发生了失真, 因此 $y_{a_1}(t)$ 部分失真.
        
        (2)
        
        由于 $x_{a_2} = \cos 2\pi t$ 的傅里叶变换为 $X_{a_2} = \pi \delta(\Omega - 5\pi) + \pi \delta(\Omega + 5\pi)$
        
        即有 $x_{a_2} = \cos 5\pi t$ 的最高频率为 $\Omega_m = 5\pi$
        
        由 $\Omega_s = 6\pi < 2 \Omega_m = 10\pi$ 与采样定理可知发生了混叠, 因此 $y_{a_2}(t)$ 失真.
        
        \item 2.
        
        (1)
        
        由线性特性和时移特性可得
        
        $\displaystyle \mathcal{F}[x(t) + x(t-t_0)] = X(j\omega) + X(j\omega)e^{-j\omega t_0}$
        
        因此可以看出最小抽样频率依然为 $\omega_0$.
        
        (2)
        
        由时域微分特性可得
        
        $\displaystyle \mathcal{F}[x'(t)] = j\omega X(j\omega)$
        
        因此可以看出最小抽样频率依然为 $\omega_0$.
        
        (3)
        
        由频域卷积特性可得
        
        $\displaystyle \mathcal{F}[x^{2}(t)] = \frac{1}{2\pi}[X(j\omega) * X(j\omega)] = \frac{1}{2\pi} \int_{-\infty}^{\infty}X(j\Omega)X(j(\omega-\Omega))\mathrm{d}\Omega$
        
        当 $-\omega_0 \le \omega \le \omega_0$ 时, $\displaystyle \frac{1}{2\pi} \int_{-\infty}^{\infty}X(j\Omega)X(j(\omega-\Omega))\mathrm{d}\Omega$ 存在不为零的可能性.
        
        因此最小抽样频率为 $2\omega_0$.
        
        (4)
        
        由频域卷积特性可得
        
        $\displaystyle \mathcal{F}[x(t)\cos \omega_0 t] = \frac{1}{2}[X(j(\omega+\omega_0)) + X(j(\omega-\omega_0))]$
        
        因此最小抽样频率为 $3\omega_0$.
			\end{itemize}
		\end{spacing}
	\end{framed}
	
	\newpage
	\section{[10pts] DFS}
	求周期为 6 的序列$ x(n)=\{\cdots, 14, 12, 10, 8, 6, 10, \cdots\} $的傅里叶级数的系数.
	
	\begin{framed}
		\begin{spacing}{1.5}
			\begin{itemize}
				\item 

        由 $\displaystyle \tilde{X}[k] = \sum_{n=0}^{N-1}\tilde{x}_{N}[n]\cdot e^{-j\frac{2\pi}{N}nk}$ 可得

        $
        \begin{aligned}
        \tilde{X}[k] & = 14 \cdot e^{-j \frac{\pi}{3} \cdot 0 \cdot k} + 12 \cdot e^{-j \frac{\pi}{3} \cdot 1 \cdot k} + 10 \cdot e^{-j \frac{\pi}{3} \cdot 2 \cdot k} + 8 \cdot e^{-j \frac{\pi}{3} \cdot 3 \cdot k}  \\
        &\quad\ + 6 \cdot e^{-j \frac{\pi}{3} \cdot 4 \cdot k} + 10 \cdot e^{-j \frac{\pi}{3} \cdot 5 \cdot k}  \\
        \end{aligned}
        $
        
        即有
        
        $
        \begin{aligned}
        \tilde{X}[0] & = 14 \cdot e^{-j \frac{\pi}{3} \cdot 0 \cdot 0} + 12 \cdot e^{-j \frac{\pi}{3} \cdot 1 \cdot 0} + 10 \cdot e^{-j \frac{\pi}{3} \cdot 2 \cdot 0} + 8 \cdot e^{-j \frac{\pi}{3} \cdot 3 \cdot 0}  \\
        &\quad\ + 6 \cdot e^{-j \frac{\pi}{3} \cdot 4 \cdot 0} + 10 \cdot e^{-j \frac{\pi}{3} \cdot 5 \cdot 0}  \\
        & = 60  \\
        \end{aligned}
        $
        
        $
        \begin{aligned}
        \tilde{X}[1] & = 14 \cdot e^{-j \frac{\pi}{3} \cdot 0 \cdot 1} + 12 \cdot e^{-j \frac{\pi}{3} \cdot 1 \cdot 1} + 10 \cdot e^{-j \frac{\pi}{3} \cdot 2 \cdot 1} + 8 \cdot e^{-j \frac{\pi}{3} \cdot 3 \cdot 1} \\
        &\quad\ + 6 \cdot e^{-j \frac{\pi}{3} \cdot 4 \cdot 1} + 10 \cdot e^{-j \frac{\pi}{3} \cdot 5 \cdot 1}  \\
        & = 14 + 12(\cos\frac{\pi}{3} - j\sin\frac{\pi}{3}) + 10(\cos\frac{2\pi}{3} - j\sin\frac{2\pi}{3}) + 8\cos\pi  \\
        &\quad\ + 6(\cos\frac{4\pi}{3} - j\sin\frac{4\pi}{3}) + 10(\cos\frac{5\pi}{3} - j\sin\frac{5\pi}{3})  \\
        & = 9 - j 3 \sqrt{3}  \\
        \end{aligned}
        $
        
        $
        \begin{aligned}
        \tilde{X}[2] & = 14 \cdot e^{-j \frac{\pi}{3} \cdot 0 \cdot 2} + 12 \cdot e^{-j \frac{\pi}{3} \cdot 1 \cdot 2} + 10 \cdot e^{-j \frac{\pi}{3} \cdot 2 \cdot 2} + 8 \cdot e^{-j \frac{\pi}{3} \cdot 3 \cdot 2} \\
        &\quad\ + 6 \cdot e^{-j \frac{\pi}{3} \cdot 4 \cdot 2} + 10 \cdot e^{-j \frac{\pi}{3} \cdot 5 \cdot 2}  \\
        & = 14 + 12(\cos\frac{2\pi}{3} - j\sin\frac{2\pi}{3}) + 10(\cos\frac{4\pi}{3} - j\sin\frac{4\pi}{3}) + 8\cos 2\pi  \\
        &\quad\ + 6(\cos\frac{8\pi}{3} - j\sin\frac{8\pi}{3}) + 10(\cos\frac{10\pi}{3} - j\sin\frac{10\pi}{3})  \\
        & = 3 + j\sqrt{3}  \\
        \end{aligned}
        $
        
        $
        \begin{aligned}
        \tilde{X}[3] & = 14 \cdot e^{-j \frac{\pi}{3} \cdot 0 \cdot 3} + 12 \cdot e^{-j \frac{\pi}{3} \cdot 1 \cdot 3} + 10 \cdot e^{-j \frac{\pi}{3} \cdot 2 \cdot 3} + 8 \cdot e^{-j \frac{\pi}{3} \cdot 3 \cdot 3} \\
        &\quad\ + 6 \cdot e^{-j \frac{\pi}{3} \cdot 4 \cdot 3} + 10 \cdot e^{-j \frac{\pi}{3} \cdot 5 \cdot 3}  \\
        & = 14 + 12(\cos\frac{3\pi}{3} - j\sin\frac{3\pi}{3}) + 10(\cos\frac{6\pi}{3} - j\sin\frac{6\pi}{3}) + 8\cos 3\pi  \\
        &\quad\ + 6(\cos\frac{12\pi}{3} - j\sin\frac{12\pi}{3}) + 10(\cos\frac{15\pi}{3} - j\sin\frac{15\pi}{3})  \\
        & = 0  \\
        \end{aligned}
        $

        \newpage
        
        $
        \begin{aligned}
        \tilde{X}[4] & = 14 \cdot e^{-j \frac{\pi}{3} \cdot 0 \cdot 4} + 12 \cdot e^{-j \frac{\pi}{3} \cdot 1 \cdot 4} + 10 \cdot e^{-j \frac{\pi}{3} \cdot 2 \cdot 4} + 8 \cdot e^{-j \frac{\pi}{3} \cdot 3 \cdot 4} \\
        &\quad\ + 6 \cdot e^{-j \frac{\pi}{3} \cdot 4 \cdot 4} + 10 \cdot e^{-j \frac{\pi}{3} \cdot 5 \cdot 4}  \\
        & = 14 + 12(\cos\frac{4\pi}{3} - j\sin\frac{4\pi}{3}) + 10(\cos\frac{8\pi}{3} - j\sin\frac{8\pi}{3}) + 8\cos 4\pi  \\
        &\quad\ + 6(\cos\frac{16\pi}{3} - j\sin\frac{16\pi}{3}) + 10(\cos\frac{20\pi}{3} - j\sin\frac{20\pi}{3})  \\
        & = 3 - j\sqrt{3}  \\
        \end{aligned}
        $
        
        $
        \begin{aligned}
        \tilde{X}[5] & = 14 \cdot e^{-j \frac{\pi}{3} \cdot 0 \cdot 5} + 12 \cdot e^{-j \frac{\pi}{3} \cdot 1 \cdot 5} + 10 \cdot e^{-j \frac{\pi}{3} \cdot 2 \cdot 5} + 8 \cdot e^{-j \frac{\pi}{3} \cdot 3 \cdot 5} \\
        &\quad\ + 6 \cdot e^{-j \frac{\pi}{3} \cdot 4 \cdot 1} + 10 \cdot e^{-j \frac{\pi}{3} \cdot 5 \cdot 1}  \\
        & = 14 + 12(\cos\frac{5\pi}{3} - j\sin\frac{5\pi}{3}) + 10(\cos\frac{10\pi}{3} - j\sin\frac{10\pi}{3}) + 8\cos 5\pi  \\
        &\quad\ + 6(\cos\frac{20\pi}{3} - j\sin\frac{20\pi}{3}) + 10(\cos\frac{25\pi}{3} - j\sin\frac{25\pi}{3})  \\
        & = 9 + j 3 \sqrt{3}  \\
        \end{aligned}
        $
			\end{itemize}
		\end{spacing}
	\end{framed}
	
	\newpage
	\section{[30pts] DTFT 及其逆变换}
	\begin{itemize}
		\item[1.] 对以下各序列, 试求其 DTFT.
		\begin{itemize}
			\item[(1)] $x(n)=(0.6)^n[u(n)-u(n-15)]$
			\item[(2)] $x(n)=n(0.8)^n[u(n)-u(n-40)]$
		\end{itemize}
		\item[2.] $ X(e^{j\omega})= \begin{cases}2j, & 0<\omega \leqslant \pi \\ -2j, & -\pi<\omega \leqslant 0\end{cases} $, 求解其逆变换 $ x(n) $
	\end{itemize}
	
	\begin{framed}
		\begin{spacing}{1.5}
			\begin{itemize}
        \item 1.

        (1)
        
        $
        \begin{aligned}
        \operatorname{DTFT}[x(n)] & = \sum_{n=-\infty}^{\infty}x(n)e^{-jn\omega} = \sum_{n=0}^{14}(0.6)^{n}e^{-jn\omega}  \\
        & = \sum_{n=0}^{14}e^{(\ln 0.6-j\omega)n} = \frac{1-e^{15(\ln 0.6-j\omega)}}{1 - e^{\ln 0.6-j\omega}}  \\
        \end{aligned}
        $
        
        (2)
        
        由于 $\displaystyle \sum_{n=0}^{N}n a^{n} = \frac{a (- (N+1) a^{N} + N a^{N + 1} + 1)}{(a - 1)^{2}}, a \neq 1$ 我们有
        
        $
        \begin{aligned}
        \operatorname{DTFT}[x(n)] & = \sum_{n=-\infty}^{\infty}x(n)e^{-jn\omega} = \sum_{n=0}^{39}n(0.8)^{n}e^{-jn\omega} = \sum_{n=0}^{39}ne^{(\ln 0.8-j\omega)n}  \\
        &= \frac{e^{\ln 0.8-j\omega} (- 40 e^{39(\ln 0.8-j\omega)} + 39 e^{40(\ln 0.8-j\omega)} + 1)}{(e^{\ln 0.8-j\omega} - 1)^{2}}  \\
        \end{aligned}
        $
        
        \item 2.
        
        $
        \begin{aligned}
        x(n) &= \operatorname{IDTFT}[X(e^{j\omega})]  \\
        &= \frac{1}{2\pi}\int_{-\pi}^{\pi}X(e^{j\omega})e^{j\omega n}\mathrm{d}\omega  \\
        &= \frac{1}{2\pi}[\int_{-\pi}^{0}-2j e^{j\omega n}\mathrm{d}\omega + \int_{0}^{\pi}2j e^{j\omega n}\mathrm{d}\omega]  \\
        &= \frac{1}{2\pi}[-\frac{2}{n}(e^{j\omega n})|_{-\pi}^{0} + \frac{2}{n} (e^{j\omega n})|_{0}^{\pi}]  \\
        &= \frac{1}{2\pi}[-\frac{2}{n}(1 - e^{j(-\pi) n}) + \frac{2}{n} (e^{j \pi n} - 1)]  \\
        &= \frac{1}{\pi n}(e^{j (-\pi n)} + e^{j \pi n} - 2)  \\
        \end{aligned}
        $
			\end{itemize}
		\end{spacing}
	\end{framed}
	
	\newpage
	\section{[30pts] DTFT 和 DFS}
	已知 $x(n)=\{2,1,4,2,3\}$
	\begin{itemize}
		\item[(1)]计算 $X({e}^{{j}\omega})=\operatorname{DTFT}[x(n)]$ 及 $X(k)=\operatorname{DFT}[x(n)]$.
		\item[(2)]将 $x(n)$ 的尾部补零, 得到 $x_0(n)=\{2, 1,4,2,3,0,0,0\}$. 计算 $X_0({e}^{j \omega})=\operatorname{DTFT}[x_0(n)]$ 及 $X_0(k)=\operatorname{DFT}[x_0(n)]$.
		\item[(3)]将 (1), (2) 的结果加以比较, 得出相应的结论.
	\end{itemize}
	
	\begin{framed}
		\begin{spacing}{1.5}
			\begin{itemize}
        \item (1)

        $\displaystyle X(e^{j\omega}) = \operatorname{DTFT}[x(n)] = 2 + e^{-j\omega} + 4e^{-j 2\omega} + 2e^{-j 3\omega} + 3e^{-j 4\omega}$
        
        $\displaystyle X(k) = \operatorname{DFT}[x(n)] = 2 + e^{-j \frac{2}{5}\pi k} + 4e^{-j \frac{4}{5}\pi k} + 2e^{-j \frac{6}{5}\pi k} + 3e^{-j \frac{8}{5}\pi k}$
        
        \item (2)
        
        $\displaystyle X_0(e^{j\omega}) = \operatorname{DTFT}[x_0(n)] = 2 + e^{-j\omega} + 4e^{-j 2\omega} + 2e^{-j 3\omega} + 3e^{-j 4\omega}$
        
        $\displaystyle X_0(k) = \operatorname{DFT}[x_0(n)] = 2 + e^{-j \frac{\pi}{4} k} + 4e^{-j \frac{\pi}{2} k} + 2e^{-j \frac{3}{4}\pi k} + 3e^{-j \pi k}$
        
        \item (3)
        
        $x(n)$ 的 DTFT $X(e^{j\omega})$ 是与采样无关的连续函数, 补零不会改变 DTFT 的频率分量, 即此处的 $X(e^{j\omega}) = X_0(e^{j\omega})$.
        
        而 $x(n)$ 的 DFT $X(k)$ 是其 DTFT $X(e^{j\omega})$ 在一个周期 $[0, 2\pi)$ 的等间距抽样, 我们有 $X(k) = X(e^{j\omega})|_{\omega=\frac{2\pi}{N}k}$, 补零会增大 $N$, 即采样更为密集了, 因此此处 $X(k) \neq X_0(k)$.
			\end{itemize}
		\end{spacing}
	\end{framed}
	
	
\end{document}