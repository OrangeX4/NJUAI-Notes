\documentclass[a4paper,UTF8]{article}
\usepackage{ctex}
\usepackage[margin=1.25in]{geometry}
\usepackage{color}
\usepackage{graphicx}
\usepackage{amssymb}
\usepackage{amsmath}
\usepackage{amsthm}
\usepackage{enumerate}
\usepackage{bm}
\usepackage{hyperref}
\usepackage{pgfplots}
\usepackage{epsfig}
\usepackage{color}
\usepackage{tcolorbox}
\usepackage{mdframed}
\usepackage{lipsum}
\usepackage{framed}
\usepackage{setspace}
\usepackage{listings}

\newmdtheoremenv{thm-box}{myThm}
\newmdtheoremenv{prop-box}{Proposition}
\newmdtheoremenv{def-box}{定义}

\setlength{\evensidemargin}{.25in}
\setlength{\textwidth}{6in}
\setlength{\topmargin}{-0.5in}
\setlength{\topmargin}{-0.5in}
% \setlength{\textheight}{9.5in}
%%%%%%%%%%%%%%%%%%此处用于设置页眉页脚%%%%%%%%%%%%%%%%%%
\usepackage{fancyhdr}                                
\usepackage{lastpage}                                           
\usepackage{layout}                                             
\footskip = 10pt 
\pagestyle{fancy}                    % 设置页眉  


%%%%%%%%%%%% 这里也填一下 %%%%%%%%%%%%%%%%%%                
\lhead{你的学号}           
%%%%%%%%%%%% 这里也填一下 %%%%%%%%%%%%%%%%%% 

           
\chead{数字信号处理}                                                
% \rhead{第\thepage/\pageref{LastPage}页} 
\rhead{作业四}                                                                                               
\cfoot{\thepage}                                                
\renewcommand{\headrulewidth}{1pt}  			%页眉线宽,设为0可以去页眉线
\setlength{\skip\footins}{0.5cm}    			%脚注与正文的距离           
\renewcommand{\footrulewidth}{0pt}  			%页脚线宽,设为0可以去页脚线

\makeatletter 									%设置双线页眉                                        
\def\headrule{{\if@fancyplain\let\headrulewidth\plainheadrulewidth\fi%
		\hrule\@height 1.0pt \@width\headwidth\vskip1pt	%上面线为1pt粗  
		\hrule\@height 0.5pt\@width\headwidth  			%下面0.5pt粗            
		\vskip-2\headrulewidth\vskip-1pt}      			%两条线的距离1pt        
	\vspace{6mm}}     								%双线与下面正文之间的垂直间距              
\makeatother  

%%%%%%%%%%%%%%%%%%%%%%%%%%%%%%%%%%%%%%%%%%%%%%
\numberwithin{equation}{section}
%\usepackage[thmmarks, amsmath, thref]{ntheorem}
\newtheorem{myThm}{myThm}
\newtheorem*{myDef}{Definition}
\newtheorem*{mySol}{Solution}
\newtheorem*{myProof}{Proof}
\newtheorem*{myRemark}{备注}
\renewcommand{\tilde}{\widetilde}
\renewcommand{\hat}{\widehat}
\newcommand{\indep}{\rotatebox[origin=c]{90}{$\models$}}
\newcommand*\diff{\mathop{}\!{d}}
\setlength{\parindent}{0pt}
\usepackage{multirow}

\lstset{language=Matlab}%这条命令可以让LaTeX排版时将Matlab关键字突出显示
\lstset{
	breaklines,%这条命令可以让LaTeX自动将长的代码行换行排版
	basicstyle=\footnotesize\ttfamily, % Standardschrift
	backgroundcolor=\color[rgb]{0.95,0.95,0.95},
	keywordstyle=\color{blue},
	commentstyle=\color{cyan},
	tabsize=4,numbers=left,
	numberstyle=\tiny,
	frame=single,
	%numbers=left, % Ort der Zeilennummern
	numberstyle=\tiny, % Stil der Zeilennummern
	%stepnumber=2, % Abstand zwischen den Zeilennummern
	numbersep=5pt, % Abstand der Nummern zum Text
	tabsize=2, % Groesse von Tabs
	extendedchars=false, %
	breaklines=true, % Zeilen werden Umgebrochen
	keywordstyle=\color{red},%这一条命令可以解决代码跨页时, 章节标题, 页眉等汉字不显示的问题
	stringstyle=\color{white}\ttfamily, % Farbe der String
	showspaces=false, % Leerzeichen anzeigen ?
	showtabs=false, % Tabs anzeigen ?
	xleftmargin=17pt,
	framexleftmargin=17pt,
	framexrightmargin=5pt,
	framexbottommargin=4pt,
	%backgroundcolor=\color{lightgray},
	showstringspaces=false % Leerzeichen in Strings anzeigen ?
}
\renewcommand{\lstlistingname}{CODE}
\lstloadlanguages{% Check Dokumentation for further languages ...
	%[Visual]Basic
	%Pascal
	%C
	Python
	%XML
	%HTML
	%Java
}


%--

%--
\begin{document}
	
	\title{数字信号处理\\
		作业四}
	\author{你的名字\, 你的学号} 
	\maketitle
	%%%%%%%% 注意: 使用XeLatex 编译可能会报错,请使用 pdfLaTex 编译 %%%%%%%
	
	\section*{作业提交注意事项}
	\begin{tcolorbox}
		\begin{enumerate}
			\item[(1)] 本次作业提交截止时间为~\textcolor{red}{\textbf{2022/12/18  23:59:59}},截止时间后不再接收作业,本次作业记零分;
			\item[(2)] 作业提交方式:使用此 \LaTeX 模板书写解答,只需提交编译生成的~pdf~文件,将~pdf~文件以 sftp 方式上传,账号为 dsp2022,密码为 12345asd!@。请远程连接 sftp://www.lamda.nju.edu.cn,提交到 /D:/courses/DSP2022/HW/HW4 路径下。
			\item[(3)] 文件命名方式:学号-姓名-作业号-v版本号, 例~ MG1900000-张三-4-v1;如果需要更改已提交的解答,请在截止时间之前提交新版本的解答,并将版本号加一;
			\item[(4)] 未按照要求提交作业,或~pdf~命名方式不正确,将会被扣除部分作业分数。
		\end{enumerate}
	\end{tcolorbox}
	
	
	\newpage
	\section{[60pts] 拉普拉斯变换}
	\begin{itemize}
		\item[1.]计算下列函数的拉普拉斯变换.
		\begin{itemize}
			\item[(1)]$ te^{-(t-2)}u(t-1) $
			\item[(2)]$ e^{-at}f\!\left(\frac{t}{a}\right) $, 已知 $\mathcal{L}[f(t)]=F(s)$
			\item[(3)]$ t^2\cos(2t) $
			\item[(4)]$ t^nu(t) $
			\item[(5)]$ f(t)=\left\{\begin{array}{ll}
				\sin (\omega t) & 0<t<\frac{T}{2},\ T=\frac{2 \pi}{\omega} \\
				0 & t \text { 为其他值 }
			\end{array} \right. $
		\end{itemize}
		
		\item[2.]求 $ \displaystyle \frac{4s+5}{s^2+5s+6} $ 的拉普拉斯反变换.
	\end{itemize}
	
	
	\begin{framed}
		\begin{spacing}{1.5}
			\begin{itemize}
				\item 你的解答。
			\end{itemize}
		\end{spacing}
	\end{framed}
	
	
	
	
	\newpage
	\section{[40pts] 拉普拉斯变换的应用}
	已知一连续时间 LTI 系统的零状态响应
	$$
	y(t)=\left(1+0.6e^{-20t}-1.6e^{-10t}\right) x(t)
	$$
	已知 $x(t)=u(t)$, 由 $s$ 域求解:
	\begin{itemize}
		\item[(1)]该系统的系统函数 $H(s)$ 并画出零极点分布图;
		\item[(2)]写出描述系统的微分方程和系统, 并求 $h(t)$;
		\item[(3)]判断系统是否因果稳定.
	\end{itemize}
	
	
\end{document}