% 请确保文件编码为utf-8,使用XeLaTex进行编译,或者通过overleaf进行编译

\documentclass[answers]{exam}  % 使用此行带有作答模块
% \documentclass{exam} % 使用此行只显示题目

\usepackage{xeCJK}
\usepackage{zhnumber}
\usepackage{graphicx}
\usepackage{hyperref}
\usepackage{amsmath}
\usepackage{booktabs}
\usepackage{enumerate}
\usepackage{amssymb}
\usepackage{listings}
\usepackage{floatrow}
\usepackage{blindtext}
\usepackage{subcaption}
\pagestyle{headandfoot}
\firstpageheadrule
\firstpageheader{南京大学}{机器学习导论}{习题三}
\runningheader{南京大学}
{机器学习导论}
{习题二}
\runningheadrule
\firstpagefooter{}{第\thepage\ 页(共\numpages 页)}{}
\runningfooter{}{第\thepage\ 页(共\numpages 页)}{}


\setlength\linefillheight{.5in}

\renewcommand{\solutiontitle}{\noindent\textbf{解:}\par\noindent}

\renewcommand{\thequestion}{\zhnum{question}}
\renewcommand{\questionlabel}{\thequestion .}
\renewcommand{\thepartno}{\arabic{partno}}
\renewcommand{\partlabel}{\thepartno .}

\lstset{language=Matlab}%这条命令可以让LaTeX排版时将Matlab关键字突出显示
\lstset{
	breaklines,%这条命令可以让LaTeX自动将长的代码行换行排版
	basicstyle=\footnotesize\ttfamily, % Standardschrift
	backgroundcolor=\color[rgb]{0.95,0.95,0.95},
	keywordstyle=\color{blue},
	commentstyle=\color{cyan},
	tabsize=4,numbers=left,
	numberstyle=\tiny,
	frame=single,
	%numbers=left, % Ort der Zeilennummern
	numberstyle=\tiny, % Stil der Zeilennummern
	%stepnumber=2, % Abstand zwischen den Zeilennummern
	numbersep=5pt, % Abstand der Nummern zum Text
	tabsize=2, % Groesse von Tabs
	extendedchars=false, %
	breaklines=true, % Zeilen werden Umgebrochen
	keywordstyle=\color{red},%这一条命令可以解决代码跨页时, 章节标题, 页眉等汉字不显示的问题
	stringstyle=\color{white}\ttfamily, % Farbe der String
	showspaces=false, % Leerzeichen anzeigen ?
	showtabs=false, % Tabs anzeigen ?
	xleftmargin=17pt,
	framexleftmargin=17pt,
	framexrightmargin=5pt,
	framexbottommargin=4pt,
	%backgroundcolor=\color{lightgray},
	showstringspaces=false % Leerzeichen in Strings anzeigen ?
}
\renewcommand{\lstlistingname}{CODE}
\lstloadlanguages{% Check Dokumentation for further languages ...
	%[Visual]Basic
	%Pascal
	%C
	Python
	%XML
	%HTML
	%Java
}
%%% load AMS-Latex Package
\usepackage{amssymb}
\usepackage{amsmath,amsfonts}
% \usepackage{amsthm}
\usepackage{amsopn}
\usepackage{bm} % bold symbol
\usepackage{multirow}



% operator in optimization
\DeclareMathOperator{\argmax}{arg\,max}
\DeclareMathOperator{\argmin}{arg\,min}
\DeclareMathOperator{\softmax}{softmax}

% special functions
\newcommand{\trace}[1]{\operatornamewithlimits{tr}\left\{#1\right\}}
\newcommand{\diag}{\operatornamewithlimits{diag}}
\newcommand{\sign}{\operatornamewithlimits{sign}}
\newcommand{\const}{\operatornamewithlimits{const}}

% special display
\newcommand{\parde}[2]{\frac{\partial #1}{\partial  #2}}

% define vector and matrix symbols
\newcommand{\vct}[1]{\boldsymbol{#1}} % vector
\newcommand{\mat}[1]{\boldsymbol{#1}} % matrix
\newcommand{\cst}[1]{\mathsf{#1}}  % constant

% shorthand
\newcommand{\vtheta}{\vct{\theta}}
\newcommand{\vmu}{\vct{\mu}}
\newcommand{\vc}{\vct{c}}
\newcommand{\vp}{\vct{p}}
\newcommand{\vq}{\vct{q}}
\newcommand{\vx}{{\vct{x}}}
\newcommand{\vy}{\vct{y}}
\newcommand{\vz}{{\vct{z}}}
\newcommand{\vu}{\vct{u}}
\newcommand{\vo}{{\vct{o}}}
\newcommand{\va}{\vct{a}}
\newcommand{\vb}{\vct{b}}
\newcommand{\ve}{\vct{e}}
\newcommand{\vr}{\vct{r}}
\newcommand{\vt}{\vct{t}}
%\newcommand{\vpsi}{\vct{\psi}}
\newcommand{\vs}{\vct{s}}
\newcommand{\vv}{\vct{v}}
\newcommand{\vw}{\vct{w}}
\newcommand{\vzero}{\vct{0}}
\newcommand{\vf}{\vct{f}}
\newcommand{\vh}{\vct{h}}
\newcommand{\vg}{\vct{g}}
\newcommand{\vphi}{\vct{\phi}}
\newcommand{\vpsi}{\vct{\psi}}
\newcommand{\ones}{\vct{1}}
\newcommand{\mU}{\mat{U}}
\newcommand{\mA}{\mat{A}}
\newcommand{\mB}{\mat{B}}
\newcommand{\mC}{\mat{C}}
\newcommand{\mD}{\mat{D}}
\newcommand{\mV}{\mat{V}}
\newcommand{\mW}{\mat{W}}
\newcommand{\mH}{\mat{H}}
\newcommand{\mI}{\mat{I}}
\newcommand{\mP}{\mat{P}}
\newcommand{\mS}{\mat{S}}
\newcommand{\mJ}{\mat{J}}
\newcommand{\mM}{\mat{M}}
\newcommand{\mT}{\mat{T}}
\newcommand{\mZ}{\mat{Z}}
\newcommand{\mO}{\mat{O}}
\newcommand{\mX}{\mat{X}}
\newcommand{\mY}{\mat{Y}}
\newcommand{\mQ}{\mat{Q}}
\newcommand{\mLambda}{\mat{\Lambda}}
% \newcommand{\mL}{\mat{L}}
\newcommand{\mmI}{\mat{I}}
\newcommand{\mK}{\mat{K}}
\newcommand{\mSigma}{\mat{\Sigma}}
\newcommand{\mOmega}{\mat{\Omega}}
\newcommand{\cC}{\cst{C}}
\newcommand{\cM}{\cst{M}}
\newcommand{\cN}{\cst{N}}
\newcommand{\cQ}{\cst{Q}}
\newcommand{\cD}{\cst{D}}
\newcommand{\cL}{\cst{L}}
\newcommand{\cK}{\cst{K}}
\newcommand{\cH}{\cst{H}}
\newcommand{\cR}{\cst{R}}
\newcommand{\cU}{\cst{U}}
\newcommand{\cS}{\cst{S}}
\newcommand{\cX}{\cst{X}}
\newcommand{\sQ}{\mathcal{Q}}
\newcommand{\sS}{\mathcal{S}}
\newcommand{\sF}{\mathcal{F}}
\newcommand{\sC}{\mathcal{C}}
\newcommand{\sX}{\mathcal{X}}
\newcommand{\sH}{\mathcal{H}}

\newcommand{\bE}{\mathbb{E}}
\newcommand{\bR}{\mathbb{R}}
\newcommand{\bH}{\mathbb{H}}

\usepackage{algorithm}  
% \usepackage{algorithmic}
%\usepackage{algorithmicx}  
\usepackage{algpseudocode}  

\floatname{algorithm}{算法}  
\renewcommand{\algorithmicrequire}{\textbf{输入:}}  
\renewcommand{\algorithmicensure}{\textbf{输出:}}  

\begin{document}
\Large
\noindent 
% 姓名学号
姓名:张三 \\
学号:1234567 \\
\begin{questions}
\question [20] \textbf{利用信息熵进行决策树划分}

\begin{enumerate}
    \item  对于不含冲突样本(即属性值相同但标记不同的样本)的训练集, 必存在与训练集一致(训练误差为0)的决策树. 如果训练集可以包含无穷多个样本, 是否一定存在与训练集一致的深度有限的决策树? 并说明理由 (仅考虑每次划分仅包含一次属性判断的决策树).
    \item 
    信息熵$\operatorname{Ent}(D)$定义如下
    \begin{align}
        \operatorname{Ent}(D)=-\sum_{k=1}^{|\mathcal{Y}|}\; p_{k} \log_{2} p_{k}\label{ch4_eq:entropy}
    \end{align}
    请证明信息熵的上下界为
     \begin{equation}
        0 \leq \operatorname{Ent}(D)\leq \log _{2}|\mathcal{Y}|
    \end{equation}
    并给出等号成立的条件. 
	\item  在ID3决策树的生成过程中, 需要计算信息增益(information gain)以生成新的结点. 设离散属性$a$有$V$个可能取值$\left\{a^{1}, a^{2}, \cdots, a^{V}\right\}$, 请考教材4.2.1节相关符号的定义证明:
    \begin{equation}
        \operatorname{Gain}(D, a)=\operatorname{Ent}(D)-\sum_{v=1}^{V} \frac{\left|D^{v}\right|}{|D|} \operatorname{Ent}\left(D^{v}\right) \geq 0
    \end{equation}
    即信息增益非负.
\end{enumerate}
	\begin{solution}
		% 请在此处作答
	\end{solution}


\question [15] \textbf{决策树划分计算} \label{ch4_prob:get_tree}

本题主要展现决策树在不同划分标准下划分的具体计算过程. 假设一个包含三个布尔属性$X, Y, Z$的属性空间, 目标函数$f=f(X, Y, Z)$作为标记空间, 它们形成的数据集如\ref{ch4_tab:bool_table}所示. 
\begin{table}[ht]
    \centering
    \caption{布尔运算样例表}\label{ch4_tab:bool_table}
    \tabcolsep 15pt
    \begin{tabular}{cccc|c||cccc|c}
        \hline 
        编号 & $X$ & $Y$ & $Z$ & $f$ & 编号 & $X$ & $Y$ & $Z$ & $f$ \\
        \hline 1 & 1 & 0 & 1 & 1 & 5 & 0 & 1 & 0 & 0\\
        2 & 1 & 1 & 0 & 0 & 6 & 0 & 0 & 1 & 0 \\
        3 & 0 & 0 & 0 & 0 & 7 & 1 & 0 & 0 & 0\\
        4 & 0 & 1 & 1 & 1 & 8 & 1 & 1 & 1 & 0\\
        \hline
    \end{tabular}
\end{table}
\begin{enumerate}
    \item 请使用信息增益作为划分准则画出决策树的生成过程. 当两个属性信息增益相同时, 依据字母顺序选择属性. 
    \item 请使用基尼指数作为划分准则画出决策树的生成过程, 当两个属性基尼指数相同时, 依据字母顺序选择属性. 
\end{enumerate}
	
	\begin{solution}
		%请在此处作答
	\end{solution}


\question [25] \textbf{决策树剪枝处理} \label{ch4_prob:prunning}

教材4.3节介绍了决策树剪枝相关内容, 给定包含5个样例的人造数据集如表\ref{ch4_tab:artificial_training_dataset}所示, 其中“爱运动”、“爱学习”是属性, “成绩高”是标记. 验证集如表\ref{ch4_tab:artificial_testing_dataset}所示. 使用信息增益为划分准则产生如图\ref{ch4_fig:decision_tree_1}所示的两棵决策树. 请回答以下问题: 
\begin{table}[!htb]
    \caption{人造数据集}
    \begin{minipage}[t]{.48\linewidth}
      \subcaption{训练集}\label{ch4_tab:artificial_training_dataset}
      \centering
        \begin{tabular}{cccc}
        \hline 编号 & 爱运动 & 爱学习 & 成绩高 \\
        \hline 1 & 是 & 是 & 是 \\
        2 & 否 & 是 & 是 \\
        3 & 是 & 否 & 否 \\
        4 & 是 & 否 & 否 \\
        5 & 否 & 否 & 是 \\
        \hline
\end{tabular}
    \end{minipage}%
    \begin{minipage}[t]{.48\linewidth}
      \centering
        \subcaption{验证集}\label{ch4_tab:artificial_testing_dataset}
        \begin{tabular}{cccc}
		\hline 编号 & 爱运动 & 爱学习 & 成绩高 \\
		\hline 6 & 是 & 是 & 是 \\
		7 & 否 & 是 & 否 \\
		8 & 是 & 否 & 否 \\
		9 & 否 & 否 & 否 \\
		\hline
		\end{tabular}
    \end{minipage} 
\end{table}
\begin{figure}[ht]
    \centering
    \includegraphics[width=0.8\textwidth]{figure/ch4_decision_tree_1.pdf}
    \caption{人造数据决策树结果}\label{ch4_fig:decision_tree_1}
\end{figure}
\begin{enumerate}
    \item 
    请验证这两棵决策树的产生过程.
    \item 
     对图\ref{ch4_fig:decision_tree_1}的结果基于该验证集进行预剪枝、后剪枝, 给出剪枝后的决策树. 
    \item
    比较预剪枝、后剪枝的结果, 每种剪枝方法在训练集、验证集上的准确率分别为多少?哪种方法拟合能力较强?
\end{enumerate}

	\begin{solution}
	    %请在此处作答
	\end{solution}

\question [20] \textbf{连续与缺失值}



\begin{enumerate}
	\item 
    考虑如表~\ref{ch4_tab:continuous_small_dataset}所示数据集,仅包含一个连续属性,请给出将该属性“数字”作为划分标准时的决策树划分结果。
    \begin{table}[h]
    \begin{center}
    \begin{tabular}{cc}
    \hline 属性 & 类别 \\
    \hline 3 & 正 \\
    4 & 负 \\
    6 & 负 \\
    9 & 正 \\
    \hline
    \end{tabular}
    \caption{连续属性数据集}\label{ch4_tab:continuous_small_dataset}
    \end{center}
    \end{table}
	\item 请阐述决策树如何处理训练时存在缺失值的情况,具体如下:考虑表~\ref{ch4_tab:bool_table}的数据集,如果发生部分缺失,变成如表~\ref{ch4_tab:missing_dataset}所示数据集(假设$X, Y, Z$只有0和1两种取值).
    \begin{table}[ht]
    \centering
    \caption{缺失数据集}\label{ch4_tab:missing_dataset}
    \begin{tabular}{ccc|c}
        \hline X & Y & Z & $f$ \\
        \hline 
        1 & 0 & - & 1\\
        - & 1 & 0 & 0\\
        0 & - & 0 & 0\\
        0 & 1 & 1 & 1\\
        - & 1 & 0 & 0\\
        0 & 0 & - & 0\\
        1 & - & 0 & 0\\
        1 & 1 & 1 & 0\\
        \hline
    \end{tabular}
\end{table}
    在这种情况下,请考虑如何处理数据中的缺失值,并结合问题~\ref{ch4_prob:get_tree}第1小问的答案进行对比,论述方法的特点以及是否有局限性。
	\item 请阐述决策树如何处理测试时存在缺失值的情况,具体如下:对于问题~\ref{ch4_prob:prunning}训练出的决策树,考虑表~\ref{ch4_tab:artificial_testing_dataset2}所示的含有缺失值的测试集,输出其标签,并论述方法的特点以及是否有局限性。
	\begin{table}[ht]
    \centering
    \caption{缺失数据集}\label{ch4_tab:artificial_testing_dataset2}
	\begin{tabular}{cccc}
		\hline 编号 & 爱运动 & 爱学习 & 成绩高 \\
		\hline 6 & 是 & - &  \\
		7 & - & 是 &  \\
		8 & 否 & - &  \\
		9 & - & 否 &  \\
		\hline
	\end{tabular}
	\end{table}
    
\end{enumerate}
	\begin{solution}
	    %请在此处作答
	\end{solution}


\question [20] \textbf{多变量决策树}

考虑如下包含10个样本的数据集, 每一列表示一个样本, 每个样本具有二个属性, 即$\vx_i = (x_{i1};\; x_{i2})$.
\begin{table}[ht]
    \begin{center}
    \begin{tabular}{ccccccccccc}
    \hline 编号 & 1 & 2 & 3 & 4 & 5 & 6 & 7 & 8 & 9 & 10 \\
    \hline $A_1$ & 24 & 53 & 23 & 25 & 32 & 52 & 22 & 43 & 52 & 48 \\ 
    $A_2$ & 40 & 52 & 25 & 77 & 48 & 110 & 38 & 44 & 27 & 65\\
    \hline 标记 & 1 & 0 & 0 & 1 & 1 & 1 & 1 & 0 & 0 & 1\\
    \hline
    \end{tabular}
    \end{center}
\end{table}
\begin{enumerate}
    \item 计算根结点的熵; 
    \item 构建分类决策树, 描述分类规则和分类误差; 
    \item 根据 $\alpha x_{1}+\beta x_{2}-1$,  构建多变量决策树,描述树的深度以及 $\alpha$ 和 $\beta$ 的值. 
\end{enumerate}
	\begin{solution}
	    %请在此处作答
	\end{solution}

\end{questions}


\end{document}