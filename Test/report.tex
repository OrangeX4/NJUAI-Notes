\documentclass[UTF8]{ctexart}
% 插入图片所使用的宏包
\usepackage{graphicx}
% 固定图片
\usepackage{float}
% 超链接和目录颜色
\usepackage[colorlinks, urlcolor=blue, linkcolor=black]{hyperref}
% 用于生成随机中文
\usepackage{zhlipsum}
% 设置section左对齐,默认是居中
% \CTEXsetup[format={\Large\bfseries}]{section}
\ctexset{ section = { format=\Large\bfseries } }
% 设置章节编号
\renewcommand\thesection{\chinese{section}、}
\renewcommand\thesubsection{\arabic{subsection}.}
\renewcommand\thesubsubsection{\alph{subsubsection})}
% 设置章节编号与章节标题的距离
\makeatletter
\renewcommand\@seccntformat[1]{%
    {\csname the#1\endcsname}\hspace{0.1em}
}
\makeatother

% 在这里设置你的标题和姓名
\title{全息投影的前世今生}
\author{赵思衡}

\begin{document}

% 标题页
\begin{titlepage}
    \newcommand{\HRule}{\rule{\linewidth}{0.5mm}}
    \center
    \quad \\[1.5cm]
    \textbf{\Huge 成像世界的奇遇结课论文 } \\[3cm]
    \makeatletter
    \HRule \\[0.8cm]
    { \huge \bfseries \@title} \\[0.4cm]
    \HRule \\[2cm]
    \begin{tabular}{rl}
        \textbf{\Large 院系:} & \textbf{\Large 人工智能学院}               \\[0.5cm]
        \textbf{\Large 姓名:} & \textbf{\Large \@author}                   \\[0.5cm]
        % 设定你的学号, 班级和邮箱
        \textbf{\Large 学号:} & \textbf{\Large 201300024}                  \\[0.5cm]
        \textbf{\Large 班级:} & \textbf{\Large 人工智能 20 级 2 班}        \\[0.5cm]
        \textbf{\Large 邮箱:} & \textbf{\Large 201300024@smail.nju.edu.cn} \\[0.5cm]
        % 默认为今天, 如果要修改为你需要的时间, 请将 \today 替换为你需要的时间
        \textbf{\Large 时间:} & \textbf{\Large \today}                     \\[2cm]
    \end{tabular}
    \vfill
\end{titlepage}

% 生成目录
\newpage
\tableofcontents
\newpage

% 加大段落间距
\setlength{\parskip}{1em}

% 主体内容
\section{全息投影简介}

目前大众所认知的全息投影有真全息投影和伪全息投影之分。

真全息投影即全息术(英语:Holography),又称全息投影、全息3D,是一种记录被摄物体反射(或透射)光波中全部信息(振幅、相位)的照相技术,而物体反射或者透射的光线可以通过记录胶片完全重建,仿佛物体就在那里一样。通过不同的方位和角度观察照片,可以看到被拍摄的物体的不同的角度,因此记录得到的像可以使人产生立体视觉。

其中“Holos”是希腊语“全部”的意思。Holography/Hologram(全息术/全息图)是匈牙利裔英国物理学家Dennis Gabor在1948年发明的,这篇开山之作发表在Nature上。

这个是真全息投影的准确定义,总结起来其实就是两点:

% 列表语法
\begin{enumerate}
    \item 一是需要裸眼,无介质,影像在空气中立体呈现;
    \item 二是可以从360度去观看立体影像的不同角度。
\end{enumerate}



\section{商用的伪全息投影技术}

既然已经知道了真全息投影的标准,那么我们就可以很容易的辨别什么是伪全息投影,什么是真全息投影了。目前谈到全息投影,我们想到最多的商用场景就是虚拟偶像演唱会,比如初音未來、洛天依。

但是很遗憾的是,它们都属于伪全息投影,并不是真正的全息投影。虽然虚拟偶像演唱会给我们呈现出了栩栩如生的立体影像,但是其必须在固定的舞台上,且要在黑暗当中才能实现,而且观众必须要从特定的角度进行观看。

以下是目前商用较多的几种伪全息投影:

\subsection{佩珀尔幻象}

虚拟偶像演唱会是怎么实现的呢?它实际上属于一种光学错觉技术,我们称之为佩珀尔幻象,在魔术表演中经常会用到。它的原理并不复杂,是利用一张半透半反的膜,也就是所谓的透明全息膜,作为介质,使得物体在膜中成了个虚像,因为是半透的,所以你可以看到膜后的景物,视觉上给人一种立体的错觉,再加上CG技术以及高亮度的灯光,这种立体影像就会给观众一种惟妙惟肖的真实感觉。

目前商用领域所谓的全息投影大多都是利用的佩珀尔幻象的原理,并不是真正的全息投影。真正的全息投影技术其实还有很多技术门槛需要被攻破,目前只能停留在实验室阶段。

全息投影技术的实现本质上和现在的电影放映技术是一样的,都是对“光”的控制。先采集所需要的内容信息,再复原这些信息,只不过电影放映技术是采集并复原平面信息,而全息投影是要采集并复原立体信息。电影放映技术是用幕布作为介质来承载内容,而目前全息投影的立体信息还没有一个成本低且稳定的介质,可以承载这些立体信息。

\subsection{旋转LED显示技术}

旋转LED显示技术看着像裹了一圈LED的风扇,转起来后会有3D视觉效果。这种技术利用了视觉暂留原理,通过LED的高速旋转来实现平面成像,但由于LED灯条在旋转时并非密不透风,观察者依然可以看到灯条后的物体,从而让观察者感觉画面悬浮在空中,实现类似3D的效果。有人叫他“3D全息风扇屏”,但这个也和“全息”没有任何关系。

\subsection{3D渲染}

3D渲染即增强现实技术(Augmented Reality,简称AR),也有对应VR虚拟实境一词的翻译称为实拟虚境或扩张现实,是指透过摄影机影像的位置及角度精算并加上图像分析技术,让屏幕上的虚拟世界能够与现实世界场景进行结合与交互的技术。3D渲染是后期人为加的3D效果,是一种视频处理手段,现实中看不到。通过后期特效与拍摄画面的合成达到效果。现场观众可以通过大屏幕观看。优点是对场地没有要求,和舞台上其他人互动性强,视角灵活,可搭配其他特效(比如变换场景之类的)。缺点自然就是无法直接看到,只能通过屏幕。


\section{全息投影技术的前沿发展}

目前全球已知的全息投影技术有三种,分别是360度全息显示屏技术、空气投影技术、激光束投射技术。其中360度全息显示屏技术最容易理解,它是将图像投射镜子上,再让镜子进行高速的旋转,从而产生3D的立体影像;空气投影技术则是利用水蒸气,将影像投射在水蒸气上,由于分子之间的震动不均衡,所以可以形成立体图像;激光束投射技术是最为复杂的,它是利用氮气和氧气在空气中散开时,混合成的气体变成灼热的浆状物质,并在空气中投射出3D影像,但这种技术显示的时间很短暂。

而现在,杨百翰大学(Brigham Young University)的研究者们,则通过激光束捕捉物理粒子,创造出了真正的能够漂浮在空气中的,动态的立体图像:想象在一个充满灰尘的房间中,用强光一照,你就能看到飞舞的灰尘反射光线,在空气中形成许多小亮点。利用激光来照射实体粒子并使其向四处反光也同理。而现在,如果我们能控制这个粒子的轨迹,并且让这个粒子在这个轨迹上进行极快的周期性运动(scan),那么此时粒子反光的轨迹就会形成一个立体图像。

这就是OTD(Optical Trap Display),\textbf{光学捕捉显示}技术。也是BYU电机工程学教授Dan Smalley和他的团队在这个项目里所使用的核心技术。

他们先使用激光束捕捉空气中的微小物理粒子,然后快速移动。当粒子被拖曳穿过空间时,可见光会通过激光束将其照亮,形成一条运动路径。而如果这时以高于眼睛的闪烁速率每秒重新绘制10次以上,就能通过视觉的持久性来形成图像。并且,在粒子的移速足够快时,其位置和颜色都可以被改变,从而形成颜色各异的动态体积图像(volume image)。

除此之外,研究团队还利用透视投影(Perspective Projection)技术,随着观看者视角的移动来修改图像的边缘及视差,在背景生成模拟的非体积图像点,以增加对图像体积或深度的感知。这时,你就完全可以环绕图像一圈,360度无死角地去观察它了。这些自由浮动的全息图像,本身是在固定体积大小的空间中,由激光束捕捉粒子构建的,所以只能生成微小的3D全息图。团队里的Wesley Rogers表示,如果要构建一座真实大小的山峰模型,那也必须有一个体积相同甚至更大的空间,来捕捉这整个空间中的粒子。这显然是不现实的,于是他们使用了一些视觉技巧,如运动视差(Motion Parallax)技术,来使场景中移动的图像在显示时,看起来比实际要大得多。而对比大多数还是要求观众盯着屏幕展现效果的3D投影,这项技术所展现的物理的,而非幻象的投影,真正做到了幻觉与真人的互动。


\section{总结}

综上所述,目前真正的全息投影大多只能限制于实验室的环境下,无法做到大规模的商用。而商用的所谓全息投影其实都不是真正的全息投影,它们无法脱离成像的介质,只能依附于某种成像镜子或膜来进行投影。而我们离真正的全息投影技术还有很远的路要走。

\begin{thebibliography}{99}  
\bibitem{ref1}Rogers W, Smalley D. Simulating virtual images in optical trap displays[J]. Scientific reports, 2021, 11(1): 1-6.
\end{thebibliography}


\end{document}